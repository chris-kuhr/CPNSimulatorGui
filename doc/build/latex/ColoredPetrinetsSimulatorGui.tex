% Generated by Sphinx.
\def\sphinxdocclass{report}
\documentclass[a4paper,10pt,english]{sphinxmanual}

\usepackage[utf8]{inputenc}
\ifdefined\DeclareUnicodeCharacter
  \DeclareUnicodeCharacter{00A0}{\nobreakspace}
\else\fi
\usepackage{cmap}
\usepackage[T1]{fontenc}
\usepackage{amsmath,amssymb}
\usepackage{babel}
\usepackage{times}
\usepackage[Bjarne]{fncychap}
\usepackage{longtable}
\usepackage{sphinx}
\usepackage{multirow}
\usepackage{eqparbox}



\addto\captionsenglish{\renewcommand{\figurename}{Fig. }}
\addto\captionsenglish{\renewcommand{\tablename}{Table }}
\SetupFloatingEnvironment{literal-block}{name=Listing }

\addto\extrasenglish{\def\pageautorefname{page}}

\setcounter{tocdepth}{1}


\title{Colored Petrinets Simulator Gui Documentation}
\date{May 06, 2016}
\release{alpha 2}
\author{Christoph Kuhr}
\newcommand{\sphinxlogo}{}
\renewcommand{\releasename}{Release}
\makeindex

\makeatletter
\def\PYG@reset{\let\PYG@it=\relax \let\PYG@bf=\relax%
    \let\PYG@ul=\relax \let\PYG@tc=\relax%
    \let\PYG@bc=\relax \let\PYG@ff=\relax}
\def\PYG@tok#1{\csname PYG@tok@#1\endcsname}
\def\PYG@toks#1+{\ifx\relax#1\empty\else%
    \PYG@tok{#1}\expandafter\PYG@toks\fi}
\def\PYG@do#1{\PYG@bc{\PYG@tc{\PYG@ul{%
    \PYG@it{\PYG@bf{\PYG@ff{#1}}}}}}}
\def\PYG#1#2{\PYG@reset\PYG@toks#1+\relax+\PYG@do{#2}}

\expandafter\def\csname PYG@tok@kc\endcsname{\let\PYG@bf=\textbf\def\PYG@tc##1{\textcolor[rgb]{0.00,0.44,0.13}{##1}}}
\expandafter\def\csname PYG@tok@nd\endcsname{\let\PYG@bf=\textbf\def\PYG@tc##1{\textcolor[rgb]{0.33,0.33,0.33}{##1}}}
\expandafter\def\csname PYG@tok@o\endcsname{\def\PYG@tc##1{\textcolor[rgb]{0.40,0.40,0.40}{##1}}}
\expandafter\def\csname PYG@tok@ch\endcsname{\let\PYG@it=\textit\def\PYG@tc##1{\textcolor[rgb]{0.25,0.50,0.56}{##1}}}
\expandafter\def\csname PYG@tok@sc\endcsname{\def\PYG@tc##1{\textcolor[rgb]{0.25,0.44,0.63}{##1}}}
\expandafter\def\csname PYG@tok@mf\endcsname{\def\PYG@tc##1{\textcolor[rgb]{0.13,0.50,0.31}{##1}}}
\expandafter\def\csname PYG@tok@c1\endcsname{\let\PYG@it=\textit\def\PYG@tc##1{\textcolor[rgb]{0.25,0.50,0.56}{##1}}}
\expandafter\def\csname PYG@tok@bp\endcsname{\def\PYG@tc##1{\textcolor[rgb]{0.00,0.44,0.13}{##1}}}
\expandafter\def\csname PYG@tok@kn\endcsname{\let\PYG@bf=\textbf\def\PYG@tc##1{\textcolor[rgb]{0.00,0.44,0.13}{##1}}}
\expandafter\def\csname PYG@tok@cm\endcsname{\let\PYG@it=\textit\def\PYG@tc##1{\textcolor[rgb]{0.25,0.50,0.56}{##1}}}
\expandafter\def\csname PYG@tok@sd\endcsname{\let\PYG@it=\textit\def\PYG@tc##1{\textcolor[rgb]{0.25,0.44,0.63}{##1}}}
\expandafter\def\csname PYG@tok@gi\endcsname{\def\PYG@tc##1{\textcolor[rgb]{0.00,0.63,0.00}{##1}}}
\expandafter\def\csname PYG@tok@k\endcsname{\let\PYG@bf=\textbf\def\PYG@tc##1{\textcolor[rgb]{0.00,0.44,0.13}{##1}}}
\expandafter\def\csname PYG@tok@na\endcsname{\def\PYG@tc##1{\textcolor[rgb]{0.25,0.44,0.63}{##1}}}
\expandafter\def\csname PYG@tok@ge\endcsname{\let\PYG@it=\textit}
\expandafter\def\csname PYG@tok@c\endcsname{\let\PYG@it=\textit\def\PYG@tc##1{\textcolor[rgb]{0.25,0.50,0.56}{##1}}}
\expandafter\def\csname PYG@tok@nn\endcsname{\let\PYG@bf=\textbf\def\PYG@tc##1{\textcolor[rgb]{0.05,0.52,0.71}{##1}}}
\expandafter\def\csname PYG@tok@go\endcsname{\def\PYG@tc##1{\textcolor[rgb]{0.20,0.20,0.20}{##1}}}
\expandafter\def\csname PYG@tok@nb\endcsname{\def\PYG@tc##1{\textcolor[rgb]{0.00,0.44,0.13}{##1}}}
\expandafter\def\csname PYG@tok@gr\endcsname{\def\PYG@tc##1{\textcolor[rgb]{1.00,0.00,0.00}{##1}}}
\expandafter\def\csname PYG@tok@mb\endcsname{\def\PYG@tc##1{\textcolor[rgb]{0.13,0.50,0.31}{##1}}}
\expandafter\def\csname PYG@tok@gu\endcsname{\let\PYG@bf=\textbf\def\PYG@tc##1{\textcolor[rgb]{0.50,0.00,0.50}{##1}}}
\expandafter\def\csname PYG@tok@gt\endcsname{\def\PYG@tc##1{\textcolor[rgb]{0.00,0.27,0.87}{##1}}}
\expandafter\def\csname PYG@tok@kt\endcsname{\def\PYG@tc##1{\textcolor[rgb]{0.56,0.13,0.00}{##1}}}
\expandafter\def\csname PYG@tok@nf\endcsname{\def\PYG@tc##1{\textcolor[rgb]{0.02,0.16,0.49}{##1}}}
\expandafter\def\csname PYG@tok@cpf\endcsname{\let\PYG@it=\textit\def\PYG@tc##1{\textcolor[rgb]{0.25,0.50,0.56}{##1}}}
\expandafter\def\csname PYG@tok@gs\endcsname{\let\PYG@bf=\textbf}
\expandafter\def\csname PYG@tok@sr\endcsname{\def\PYG@tc##1{\textcolor[rgb]{0.14,0.33,0.53}{##1}}}
\expandafter\def\csname PYG@tok@ss\endcsname{\def\PYG@tc##1{\textcolor[rgb]{0.32,0.47,0.09}{##1}}}
\expandafter\def\csname PYG@tok@kd\endcsname{\let\PYG@bf=\textbf\def\PYG@tc##1{\textcolor[rgb]{0.00,0.44,0.13}{##1}}}
\expandafter\def\csname PYG@tok@il\endcsname{\def\PYG@tc##1{\textcolor[rgb]{0.13,0.50,0.31}{##1}}}
\expandafter\def\csname PYG@tok@mo\endcsname{\def\PYG@tc##1{\textcolor[rgb]{0.13,0.50,0.31}{##1}}}
\expandafter\def\csname PYG@tok@gp\endcsname{\let\PYG@bf=\textbf\def\PYG@tc##1{\textcolor[rgb]{0.78,0.36,0.04}{##1}}}
\expandafter\def\csname PYG@tok@nl\endcsname{\let\PYG@bf=\textbf\def\PYG@tc##1{\textcolor[rgb]{0.00,0.13,0.44}{##1}}}
\expandafter\def\csname PYG@tok@nt\endcsname{\let\PYG@bf=\textbf\def\PYG@tc##1{\textcolor[rgb]{0.02,0.16,0.45}{##1}}}
\expandafter\def\csname PYG@tok@err\endcsname{\def\PYG@bc##1{\setlength{\fboxsep}{0pt}\fcolorbox[rgb]{1.00,0.00,0.00}{1,1,1}{\strut ##1}}}
\expandafter\def\csname PYG@tok@gd\endcsname{\def\PYG@tc##1{\textcolor[rgb]{0.63,0.00,0.00}{##1}}}
\expandafter\def\csname PYG@tok@se\endcsname{\let\PYG@bf=\textbf\def\PYG@tc##1{\textcolor[rgb]{0.25,0.44,0.63}{##1}}}
\expandafter\def\csname PYG@tok@si\endcsname{\let\PYG@it=\textit\def\PYG@tc##1{\textcolor[rgb]{0.44,0.63,0.82}{##1}}}
\expandafter\def\csname PYG@tok@vc\endcsname{\def\PYG@tc##1{\textcolor[rgb]{0.73,0.38,0.84}{##1}}}
\expandafter\def\csname PYG@tok@vg\endcsname{\def\PYG@tc##1{\textcolor[rgb]{0.73,0.38,0.84}{##1}}}
\expandafter\def\csname PYG@tok@m\endcsname{\def\PYG@tc##1{\textcolor[rgb]{0.13,0.50,0.31}{##1}}}
\expandafter\def\csname PYG@tok@no\endcsname{\def\PYG@tc##1{\textcolor[rgb]{0.38,0.68,0.84}{##1}}}
\expandafter\def\csname PYG@tok@nc\endcsname{\let\PYG@bf=\textbf\def\PYG@tc##1{\textcolor[rgb]{0.05,0.52,0.71}{##1}}}
\expandafter\def\csname PYG@tok@mi\endcsname{\def\PYG@tc##1{\textcolor[rgb]{0.13,0.50,0.31}{##1}}}
\expandafter\def\csname PYG@tok@s2\endcsname{\def\PYG@tc##1{\textcolor[rgb]{0.25,0.44,0.63}{##1}}}
\expandafter\def\csname PYG@tok@kr\endcsname{\let\PYG@bf=\textbf\def\PYG@tc##1{\textcolor[rgb]{0.00,0.44,0.13}{##1}}}
\expandafter\def\csname PYG@tok@cp\endcsname{\def\PYG@tc##1{\textcolor[rgb]{0.00,0.44,0.13}{##1}}}
\expandafter\def\csname PYG@tok@ni\endcsname{\let\PYG@bf=\textbf\def\PYG@tc##1{\textcolor[rgb]{0.84,0.33,0.22}{##1}}}
\expandafter\def\csname PYG@tok@ow\endcsname{\let\PYG@bf=\textbf\def\PYG@tc##1{\textcolor[rgb]{0.00,0.44,0.13}{##1}}}
\expandafter\def\csname PYG@tok@nv\endcsname{\def\PYG@tc##1{\textcolor[rgb]{0.73,0.38,0.84}{##1}}}
\expandafter\def\csname PYG@tok@s1\endcsname{\def\PYG@tc##1{\textcolor[rgb]{0.25,0.44,0.63}{##1}}}
\expandafter\def\csname PYG@tok@gh\endcsname{\let\PYG@bf=\textbf\def\PYG@tc##1{\textcolor[rgb]{0.00,0.00,0.50}{##1}}}
\expandafter\def\csname PYG@tok@sh\endcsname{\def\PYG@tc##1{\textcolor[rgb]{0.25,0.44,0.63}{##1}}}
\expandafter\def\csname PYG@tok@mh\endcsname{\def\PYG@tc##1{\textcolor[rgb]{0.13,0.50,0.31}{##1}}}
\expandafter\def\csname PYG@tok@sx\endcsname{\def\PYG@tc##1{\textcolor[rgb]{0.78,0.36,0.04}{##1}}}
\expandafter\def\csname PYG@tok@kp\endcsname{\def\PYG@tc##1{\textcolor[rgb]{0.00,0.44,0.13}{##1}}}
\expandafter\def\csname PYG@tok@cs\endcsname{\def\PYG@tc##1{\textcolor[rgb]{0.25,0.50,0.56}{##1}}\def\PYG@bc##1{\setlength{\fboxsep}{0pt}\colorbox[rgb]{1.00,0.94,0.94}{\strut ##1}}}
\expandafter\def\csname PYG@tok@ne\endcsname{\def\PYG@tc##1{\textcolor[rgb]{0.00,0.44,0.13}{##1}}}
\expandafter\def\csname PYG@tok@s\endcsname{\def\PYG@tc##1{\textcolor[rgb]{0.25,0.44,0.63}{##1}}}
\expandafter\def\csname PYG@tok@w\endcsname{\def\PYG@tc##1{\textcolor[rgb]{0.73,0.73,0.73}{##1}}}
\expandafter\def\csname PYG@tok@vi\endcsname{\def\PYG@tc##1{\textcolor[rgb]{0.73,0.38,0.84}{##1}}}
\expandafter\def\csname PYG@tok@sb\endcsname{\def\PYG@tc##1{\textcolor[rgb]{0.25,0.44,0.63}{##1}}}

\def\PYGZbs{\char`\\}
\def\PYGZus{\char`\_}
\def\PYGZob{\char`\{}
\def\PYGZcb{\char`\}}
\def\PYGZca{\char`\^}
\def\PYGZam{\char`\&}
\def\PYGZlt{\char`\<}
\def\PYGZgt{\char`\>}
\def\PYGZsh{\char`\#}
\def\PYGZpc{\char`\%}
\def\PYGZdl{\char`\$}
\def\PYGZhy{\char`\-}
\def\PYGZsq{\char`\'}
\def\PYGZdq{\char`\"}
\def\PYGZti{\char`\~}
% for compatibility with earlier versions
\def\PYGZat{@}
\def\PYGZlb{[}
\def\PYGZrb{]}
\makeatother

\renewcommand\PYGZsq{\textquotesingle}

\begin{document}

\maketitle
\tableofcontents
\phantomsection\label{index::doc}


Contents:


\chapter{gui}
\label{gui_link:gui}\label{gui_link:module-gui.MainWindow}\label{gui_link:welcome-to-colored-petrinets-simulator-gui-s-documentation}\label{gui_link::doc}\index{gui.MainWindow (module)}\index{MainWindow (class in gui.MainWindow)}

\begin{fulllineitems}
\phantomsection\label{gui_link:gui.MainWindow.MainWindow}\pysiglinewithargsret{\strong{class }\code{gui.MainWindow.}\bfcode{MainWindow}}{\emph{workingDir}}{}
Bases: \code{PyQt4.QtGui.QMainWindow}

The MainWindow is loaded by the application main loop.
\begin{quote}\begin{description}
\item[{Member workingDir}] \leavevmode
Working directory.

\item[{Member editors}] \leavevmode
List of all editors: \titleref{gui.DiagramEditor}.

\item[{Member editorwidgets}] \leavevmode
List of all editor widgets.

\item[{Member logWidget}] \leavevmode
Log widget.

\item[{Member tabWidget}] \leavevmode
Widget containing \titleref{editors{[}0{]}}, and the \titleref{logWidget}.

\item[{Member simulator}] \leavevmode
\titleref{model.CPNSimulator} instance.

\item[{Member timer}] \leavevmode
Simulation timer.

\end{description}\end{quote}
\index{addSubnetEditor() (gui.MainWindow.MainWindow method)}

\begin{fulllineitems}
\phantomsection\label{gui_link:gui.MainWindow.MainWindow.addSubnetEditor}\pysiglinewithargsret{\bfcode{addSubnetEditor}}{\emph{subnet}}{}
Create a subnet editor.

@param subnet: The string name of the substitution transition, name of the subnet

\end{fulllineitems}

\index{closeEvent() (gui.MainWindow.MainWindow method)}

\begin{fulllineitems}
\phantomsection\label{gui_link:gui.MainWindow.MainWindow.closeEvent}\pysiglinewithargsret{\bfcode{closeEvent}}{\emph{event}}{}
Close window event.
\begin{quote}\begin{description}
\item[{Parameters}] \leavevmode
\textbf{\texttt{event}} -- \titleref{QtGui.closeEvent}.

\end{description}\end{quote}

\end{fulllineitems}

\index{createItemsAssignToEditor() (gui.MainWindow.MainWindow method)}

\begin{fulllineitems}
\phantomsection\label{gui_link:gui.MainWindow.MainWindow.createItemsAssignToEditor}\pysiglinewithargsret{\bfcode{createItemsAssignToEditor}}{\emph{editor}, \emph{subnet}, \emph{tmpSubnets}, \emph{importSubnet=False}}{}
Assign places, transition and connections to subnet editors
\begin{quote}\begin{description}
\item[{Parameters}] \leavevmode\begin{itemize}
\item {} 
\textbf{\texttt{editor}} -- \titleref{gui.DiagramEditor} subnet.

\item {} 
\textbf{\texttt{subnet}} -- Subnet string name.

\item {} 
\textbf{\texttt{tmpSubnets}} -- List of subnet names and corresponding items {[}''subnet'',{[}items{]}{]}.

\item {} 
\textbf{\texttt{importSubnet}} -- Flag determining, whethera new net is loaded or a subnet is imported.

\end{itemize}

\item[{Return subnetConnections}] \leavevmode
List of connection in subnet for later inter subnet processing.

\end{description}\end{quote}

\end{fulllineitems}

\index{createSubnetItemLists() (gui.MainWindow.MainWindow method)}

\begin{fulllineitems}
\phantomsection\label{gui_link:gui.MainWindow.MainWindow.createSubnetItemLists}\pysiglinewithargsret{\bfcode{createSubnetItemLists}}{\emph{tmpSubnets}, \emph{serPlaces}, \emph{serTransitions}, \emph{serConnections}}{}
Assign subnet to items.
\begin{quote}\begin{description}
\item[{Parameters}] \leavevmode\begin{itemize}
\item {} 
\textbf{\texttt{tmpSubnets}} -- List of subnet names

\item {} 
\textbf{\texttt{serPlaces}} -- List with parameters for the Place generation: {[}{[}uniqueName, name, portClone, port, {[}pos{]}, {[}initMarking{]}{]},...{]}.

\item {} 
\textbf{\texttt{serTransitions}} -- List with parameters for the Transition generation: {[}{[}uniqueName, name, {[}pos{]}, guardExpression, subnet{]},...{]}.

\item {} 
\textbf{\texttt{serConnections}} -- List with parameters for the Connection generation: {[}{[}uniqueNameSRC, uniqueNameDST, name, sourceConnector, destinationConnector{]},...{]}.

\end{itemize}

\item[{Return newTmpSubnets}] \leavevmode
List of subnet names and corresponding items {[}''subnet'',{[}items{]}{]}.

\end{description}\end{quote}

\end{fulllineitems}

\index{export\_Step\_as\_SVG() (gui.MainWindow.MainWindow method)}

\begin{fulllineitems}
\phantomsection\label{gui_link:gui.MainWindow.MainWindow.export_Step_as_SVG}\pysiglinewithargsret{\bfcode{export\_Step\_as\_SVG}}{}{}
Export Petrinet and subnets to seperate SVG files

\end{fulllineitems}

\index{importSubnet() (gui.MainWindow.MainWindow method)}

\begin{fulllineitems}
\phantomsection\label{gui_link:gui.MainWindow.MainWindow.importSubnet}\pysiglinewithargsret{\bfcode{importSubnet}}{\emph{transition}}{}
Import subnet into editor.
\begin{quote}\begin{description}
\item[{Parameters}] \leavevmode
\textbf{\texttt{transition}} -- Substitution transition \titleref{model.TransitionItem}.

\end{description}\end{quote}

\end{fulllineitems}

\index{initNetFromFile() (gui.MainWindow.MainWindow method)}

\begin{fulllineitems}
\phantomsection\label{gui_link:gui.MainWindow.MainWindow.initNetFromFile}\pysiglinewithargsret{\bfcode{initNetFromFile}}{\emph{editor}, \emph{subnetList}}{}
Create all items for a subnet.
\begin{quote}\begin{description}
\item[{Parameters}] \leavevmode\begin{itemize}
\item {} 
\textbf{\texttt{editor}} -- \titleref{gui.DiagramEditor} subnet.

\item {} 
\textbf{\texttt{subnetList}} -- ``subnet'',{[}items{]}.

\end{itemize}

\item[{Return places, transitions, connectionsInSubnet}] \leavevmode
List of visual places, List of visual Transitions, list of visualk connections.

\end{description}\end{quote}

\end{fulllineitems}

\index{loadNet() (gui.MainWindow.MainWindow method)}

\begin{fulllineitems}
\phantomsection\label{gui_link:gui.MainWindow.MainWindow.loadNet}\pysiglinewithargsret{\bfcode{loadNet}}{\emph{editor=None}, \emph{filename=None}, \emph{subnet=None}, \emph{importSubnet=False}}{}
Load a net with subnets or import subnets from file.
\begin{quote}\begin{description}
\item[{Parameters}] \leavevmode\begin{itemize}
\item {} 
\textbf{\texttt{editor}} -- \titleref{gui.DiagramEditor} to fill with the subnet. \titleref{editors{[}0{]}} if editor=none

\item {} 
\textbf{\texttt{filename}} -- The filename that is used when a new net is loaded.

\item {} 
\textbf{\texttt{subnet}} -- Subnet string name.

\item {} 
\textbf{\texttt{importSubnet}} -- Flag determining, whethera new net is loaded or a subnet is imported.

\end{itemize}

\end{description}\end{quote}

\end{fulllineitems}

\index{lookupSubstitutionConnectors() (gui.MainWindow.MainWindow method)}

\begin{fulllineitems}
\phantomsection\label{gui_link:gui.MainWindow.MainWindow.lookupSubstitutionConnectors}\pysiglinewithargsret{\bfcode{lookupSubstitutionConnectors}}{\emph{subnetConnections}}{}
Lookup and substitute connectors.
\begin{quote}\begin{description}
\item[{Parameters}] \leavevmode
\textbf{\texttt{subnetConnections}} -- List of connection in subnet for inter subnet processing.

\end{description}\end{quote}

\end{fulllineitems}

\index{newNet() (gui.MainWindow.MainWindow method)}

\begin{fulllineitems}
\phantomsection\label{gui_link:gui.MainWindow.MainWindow.newNet}\pysiglinewithargsret{\bfcode{newNet}}{\emph{init=False}}{}
Create a new Petrinet.

This method resets the main editor widget. It deletes 
the previously loaded Petrinet, as well as any hierarchical subnet.

@param init: Load on startup?

\end{fulllineitems}

\index{openSubnet() (gui.MainWindow.MainWindow method)}

\begin{fulllineitems}
\phantomsection\label{gui_link:gui.MainWindow.MainWindow.openSubnet}\pysiglinewithargsret{\bfcode{openSubnet}}{\emph{transition}}{}
Open subnet editor.
\begin{quote}\begin{description}
\item[{Parameters}] \leavevmode
\textbf{\texttt{transition}} -- Substitution transition \titleref{model.TransitionItem}.

\end{description}\end{quote}

\end{fulllineitems}

\index{saveNet() (gui.MainWindow.MainWindow method)}

\begin{fulllineitems}
\phantomsection\label{gui_link:gui.MainWindow.MainWindow.saveNet}\pysiglinewithargsret{\bfcode{saveNet}}{}{}
Save Petrinet with respect to subnets to XML file.

\end{fulllineitems}

\index{setArc() (gui.MainWindow.MainWindow method)}

\begin{fulllineitems}
\phantomsection\label{gui_link:gui.MainWindow.MainWindow.setArc}\pysiglinewithargsret{\bfcode{setArc}}{\emph{editor}, \emph{srcConnector}, \emph{dstConnector}, \emph{itemName}}{}
Create arcs based on new uniqueNames.
\begin{quote}\begin{description}
\item[{Parameters}] \leavevmode\begin{itemize}
\item {} 
\textbf{\texttt{editor}} -- \titleref{gui.DiagramEditor} subnet.

\item {} 
\textbf{\texttt{srcConnector}} -- Source \titleref{model.AbstractItem.Connector}.

\item {} 
\textbf{\texttt{dstConnector}} -- Destination \titleref{model.AbstractItem.Connector}.

\item {} 
\textbf{\texttt{itemName}} -- Annotation text.

\end{itemize}

\item[{Return newArc}] \leavevmode
Newly created \titleref{model.ArcItem}.

\end{description}\end{quote}

\end{fulllineitems}


\end{fulllineitems}

\index{bin\_() (in module gui.MainWindow)}

\begin{fulllineitems}
\phantomsection\label{gui_link:gui.MainWindow.bin_}\pysiglinewithargsret{\code{gui.MainWindow.}\bfcode{bin\_}}{\emph{QTextStream}}{{ $\rightarrow$ QTextStream}}
\end{fulllineitems}

\index{hex\_() (in module gui.MainWindow)}

\begin{fulllineitems}
\phantomsection\label{gui_link:gui.MainWindow.hex_}\pysiglinewithargsret{\code{gui.MainWindow.}\bfcode{hex\_}}{\emph{QTextStream}}{{ $\rightarrow$ QTextStream}}
\end{fulllineitems}

\index{oct\_() (in module gui.MainWindow)}

\begin{fulllineitems}
\phantomsection\label{gui_link:gui.MainWindow.oct_}\pysiglinewithargsret{\code{gui.MainWindow.}\bfcode{oct\_}}{\emph{QTextStream}}{{ $\rightarrow$ QTextStream}}
\end{fulllineitems}

\phantomsection\label{gui_link:module-gui.DiagramEditor}\index{gui.DiagramEditor (module)}\index{DiagramEditor (class in gui.DiagramEditor)}

\begin{fulllineitems}
\phantomsection\label{gui_link:gui.DiagramEditor.DiagramEditor}\pysiglinewithargsret{\strong{class }\code{gui.DiagramEditor.}\bfcode{DiagramEditor}}{\emph{mainWindow=None}, \emph{parent=None}, \emph{workingDir='`}, \emph{subnet=None}}{}
Bases: \code{PyQt4.QtGui.QWidget}

Editor widget, containing the \titleref{libraryModelView}, the \titleref{colorSetListView} and the \titleref{gui.DiagramScene}.
\begin{quote}\begin{description}
\item[{Member mainWindow}] \leavevmode
\titleref{gui.MainWindow}. Main application window.

\item[{Member parent}] \leavevmode
Parent widget.

\item[{Member workingDir}] \leavevmode
Working directory.

\item[{Member subnet}] \leavevmode
Name of visualized subnet, also window title.

\item[{Member colorListItems}] \leavevmode
Defined colors.

\item[{Member libItems}] \leavevmode
Visual library items, CPN elements.

\item[{Member portConnections}] \leavevmode
List of connections between portClone port places and substitution transitions.

\item[{Member visualPlaces}] \leavevmode
List of all \titleref{model.PlaceItem{}`s shown in this {}`gui.DiagramScene}.

\item[{Member visualTransitions}] \leavevmode
List of all \titleref{model.TransitionItem{}`s shown in this {}`gui.DiagramScene}.

\item[{Member visualConnectionList}] \leavevmode
List of all \titleref{model.ArcItem{}`s shown in this {}`gui.DiagramScene}.

\item[{Member libraryModel}] \leavevmode
Library model for visual CPN elements.

\item[{Member mouseScreenPos}] \leavevmode
Recorded mouse position.

\item[{Member diagramScene}] \leavevmode
Contained \titleref{gui.DiagramScene}.

\item[{Member diagramView}] \leavevmode
\titleref{gui.DiagramView} controlling \titleref{gui.DiagramScene}.

\item[{Member startedArc}] \leavevmode
Status, determining whether the creation of a connection has started.

\item[{Member showCanvasInfos}] \leavevmode
Status variable for activation of the tooltip \titleref{model.AbstractItem.DescriptionCanvas} s.

\end{description}\end{quote}
\index{createIcon() (gui.DiagramEditor.DiagramEditor method)}

\begin{fulllineitems}
\phantomsection\label{gui_link:gui.DiagramEditor.DiagramEditor.createIcon}\pysiglinewithargsret{\bfcode{createIcon}}{\emph{nodeType}}{}
Create Icons for \titleref{libraryModel}.
\begin{quote}\begin{description}
\item[{Parameters}] \leavevmode
\textbf{\texttt{nodeType}} -- Type of icon to create.

\end{description}\end{quote}

\end{fulllineitems}

\index{defineNewToken() (gui.DiagramEditor.DiagramEditor method)}

\begin{fulllineitems}
\phantomsection\label{gui_link:gui.DiagramEditor.DiagramEditor.defineNewToken}\pysiglinewithargsret{\bfcode{defineNewToken}}{}{}
Define a new color set.

\end{fulllineitems}

\index{deleteArc() (gui.DiagramEditor.DiagramEditor method)}

\begin{fulllineitems}
\phantomsection\label{gui_link:gui.DiagramEditor.DiagramEditor.deleteArc}\pysiglinewithargsret{\bfcode{deleteArc}}{\emph{editor}, \emph{connection}}{}
Delete arc.
\begin{quote}\begin{description}
\item[{Parameters}] \leavevmode\begin{itemize}
\item {} 
\textbf{\texttt{editor}} -- \titleref{gui.DiagramEditor} containing the \titleref{connection}.

\item {} 
\textbf{\texttt{connection}} -- \titleref{model.ArcItem} for deletion.

\end{itemize}

\end{description}\end{quote}

\end{fulllineitems}

\index{deleteItems() (gui.DiagramEditor.DiagramEditor method)}

\begin{fulllineitems}
\phantomsection\label{gui_link:gui.DiagramEditor.DiagramEditor.deleteItems}\pysiglinewithargsret{\bfcode{deleteItems}}{\emph{editor}, \emph{items}}{}
Delete items.
\begin{quote}\begin{description}
\item[{Parameters}] \leavevmode\begin{itemize}
\item {} 
\textbf{\texttt{editor}} -- \titleref{gui.DiagramEditor} containing the \titleref{items}.

\item {} 
\textbf{\texttt{items}} -- List with items (selection) to delete.

\end{itemize}

\end{description}\end{quote}

\end{fulllineitems}

\index{deletePlace() (gui.DiagramEditor.DiagramEditor method)}

\begin{fulllineitems}
\phantomsection\label{gui_link:gui.DiagramEditor.DiagramEditor.deletePlace}\pysiglinewithargsret{\bfcode{deletePlace}}{\emph{editor}, \emph{place}, \emph{portClone=False}}{}
Delete places, port places and port clone places.
\begin{quote}\begin{description}
\item[{Parameters}] \leavevmode\begin{itemize}
\item {} 
\textbf{\texttt{editor}} -- \titleref{gui.DiagramEditor} containing the \titleref{place}.

\item {} 
\textbf{\texttt{place}} -- \titleref{model.PlaceItem} for deletion.

\item {} 
\textbf{\texttt{portClone}} -- Flag determining whether a port clone place shall be deleted.

\end{itemize}

\end{description}\end{quote}

\end{fulllineitems}

\index{deleteSubnet() (gui.DiagramEditor.DiagramEditor method)}

\begin{fulllineitems}
\phantomsection\label{gui_link:gui.DiagramEditor.DiagramEditor.deleteSubnet}\pysiglinewithargsret{\bfcode{deleteSubnet}}{\emph{editor}}{}
Recursivly delete subnet editor and their content.
\begin{quote}\begin{description}
\item[{Parameters}] \leavevmode
\textbf{\texttt{editor}} -- Root \titleref{gui.DiagramEditor} for deletion.

\end{description}\end{quote}

\end{fulllineitems}

\index{deleteToken() (gui.DiagramEditor.DiagramEditor method)}

\begin{fulllineitems}
\phantomsection\label{gui_link:gui.DiagramEditor.DiagramEditor.deleteToken}\pysiglinewithargsret{\bfcode{deleteToken}}{\emph{editor}, \emph{token}}{}
Delete a token.
\begin{quote}\begin{description}
\item[{Parameters}] \leavevmode\begin{itemize}
\item {} 
\textbf{\texttt{editor}} -- \titleref{gui.DiagramEditor} containing the \titleref{place}.

\item {} 
\textbf{\texttt{token}} -- \titleref{model.TokenItem} for deletion.

\end{itemize}

\end{description}\end{quote}

\end{fulllineitems}

\index{deleteTransition() (gui.DiagramEditor.DiagramEditor method)}

\begin{fulllineitems}
\phantomsection\label{gui_link:gui.DiagramEditor.DiagramEditor.deleteTransition}\pysiglinewithargsret{\bfcode{deleteTransition}}{\emph{editor}, \emph{transition}}{}
Delete transitions, substitution transitions and their subnets.
\begin{quote}\begin{description}
\item[{Parameters}] \leavevmode\begin{itemize}
\item {} 
\textbf{\texttt{editor}} -- \titleref{gui.DiagramEditor} containing the \titleref{transition}.

\item {} 
\textbf{\texttt{transition}} -- \titleref{model.TransitionItem} for deletion.

\end{itemize}

\end{description}\end{quote}

\end{fulllineitems}

\index{keyPressEvent() (gui.DiagramEditor.DiagramEditor method)}

\begin{fulllineitems}
\phantomsection\label{gui_link:gui.DiagramEditor.DiagramEditor.keyPressEvent}\pysiglinewithargsret{\bfcode{keyPressEvent}}{\emph{event}}{}
Callback method, when a key is pressed on the keyboard.
\begin{quote}\begin{description}
\item[{Parameters}] \leavevmode
\textbf{\texttt{event}} -- \titleref{QtGui.keyPressEvent}.

\end{description}\end{quote}

\end{fulllineitems}

\index{newToken() (gui.DiagramEditor.DiagramEditor method)}

\begin{fulllineitems}
\phantomsection\label{gui_link:gui.DiagramEditor.DiagramEditor.newToken}\pysiglinewithargsret{\bfcode{newToken}}{\emph{item}}{}
Callback method, when \titleref{add Token} is clicked in \titleref{colorListView}.
\begin{quote}\begin{description}
\item[{Parameters}] \leavevmode
\textbf{\texttt{item}} -- \titleref{colorListView} item.

\end{description}\end{quote}

\end{fulllineitems}

\index{sceneMouseMoveEvent() (gui.DiagramEditor.DiagramEditor method)}

\begin{fulllineitems}
\phantomsection\label{gui_link:gui.DiagramEditor.DiagramEditor.sceneMouseMoveEvent}\pysiglinewithargsret{\bfcode{sceneMouseMoveEvent}}{\emph{event}}{}
Catch sceneMouseMoveEvent during arc creation.
\begin{quote}\begin{description}
\item[{Parameters}] \leavevmode
\textbf{\texttt{event}} -- \titleref{QtGui.sceneMouseMoveEvent}.

\end{description}\end{quote}

\end{fulllineitems}

\index{sceneMouseReleaseEvent() (gui.DiagramEditor.DiagramEditor method)}

\begin{fulllineitems}
\phantomsection\label{gui_link:gui.DiagramEditor.DiagramEditor.sceneMouseReleaseEvent}\pysiglinewithargsret{\bfcode{sceneMouseReleaseEvent}}{\emph{event}}{}
Catch \titleref{QtGui.sceneMouseReleaseEvent} on arc completion.
\begin{quote}\begin{description}
\item[{Parameters}] \leavevmode
\textbf{\texttt{event}} -- \titleref{QtGui.sceneMouseReleaseEvent}.

\end{description}\end{quote}

\end{fulllineitems}

\index{setPortConnection() (gui.DiagramEditor.DiagramEditor method)}

\begin{fulllineitems}
\phantomsection\label{gui_link:gui.DiagramEditor.DiagramEditor.setPortConnection}\pysiglinewithargsret{\bfcode{setPortConnection}}{\emph{portPlaceClone}, \emph{transition}}{}
Set a connection between a port place clone and a substitution transition.
\begin{quote}\begin{description}
\item[{Parameters}] \leavevmode\begin{itemize}
\item {} 
\textbf{\texttt{portPlaceClone}} -- Port place clone \titleref{model.PlaceItem}.

\item {} 
\textbf{\texttt{transition}} -- Substitutaion transition \titleref{model.TransitionItem}.

\end{itemize}

\end{description}\end{quote}

\end{fulllineitems}

\index{setTokenForPlace() (gui.DiagramEditor.DiagramEditor method)}

\begin{fulllineitems}
\phantomsection\label{gui_link:gui.DiagramEditor.DiagramEditor.setTokenForPlace}\pysiglinewithargsret{\bfcode{setTokenForPlace}}{\emph{place}, \emph{actualMarking}}{}
Show the tokens contained in place.
\begin{quote}\begin{description}
\item[{Parameters}] \leavevmode\begin{itemize}
\item {} 
\textbf{\texttt{place}} -- \titleref{model.PlaceItem} to show the \titleref{model.TokenItem} for.

\item {} 
\textbf{\texttt{actualMarking}} -- SNAKES Marking to determine the tokens for \titleref{place}.

\end{itemize}

\end{description}\end{quote}

\end{fulllineitems}

\index{setTokens() (gui.DiagramEditor.DiagramEditor method)}

\begin{fulllineitems}
\phantomsection\label{gui_link:gui.DiagramEditor.DiagramEditor.setTokens}\pysiglinewithargsret{\bfcode{setTokens}}{\emph{actualMarking}}{}
Set tokens for \titleref{actualMarking}.
\begin{quote}\begin{description}
\item[{Parameters}] \leavevmode
\textbf{\texttt{actualMarking}} -- SNAKES Marking to determine the tokens for \titleref{place}.

\end{description}\end{quote}

\end{fulllineitems}

\index{shortcutCreateNode() (gui.DiagramEditor.DiagramEditor method)}

\begin{fulllineitems}
\phantomsection\label{gui_link:gui.DiagramEditor.DiagramEditor.shortcutCreateNode}\pysiglinewithargsret{\bfcode{shortcutCreateNode}}{}{}
Create CPN elements with keyboard shortcuts.

\end{fulllineitems}

\index{startArc() (gui.DiagramEditor.DiagramEditor method)}

\begin{fulllineitems}
\phantomsection\label{gui_link:gui.DiagramEditor.DiagramEditor.startArc}\pysiglinewithargsret{\bfcode{startArc}}{\emph{nodeItem}}{}
Start an arc conneciton.
\begin{quote}\begin{description}
\item[{Parameters}] \leavevmode
\textbf{\texttt{nodeItem}} -- \titleref{model.AbstractItem} source element.

\end{description}\end{quote}

\end{fulllineitems}

\index{validConnection() (gui.DiagramEditor.DiagramEditor method)}

\begin{fulllineitems}
\phantomsection\label{gui_link:gui.DiagramEditor.DiagramEditor.validConnection}\pysiglinewithargsret{\bfcode{validConnection}}{}{}
Callback method, when a valid arc connection was created.

\end{fulllineitems}

\index{wheelEvent() (gui.DiagramEditor.DiagramEditor method)}

\begin{fulllineitems}
\phantomsection\label{gui_link:gui.DiagramEditor.DiagramEditor.wheelEvent}\pysiglinewithargsret{\bfcode{wheelEvent}}{\emph{event}}{}
Callback method, when the mouse wheel is used.
\begin{quote}\begin{description}
\item[{Parameters}] \leavevmode
\textbf{\texttt{event}} -- \titleref{QtGui.wheelEvent}.

\end{description}\end{quote}

\end{fulllineitems}


\end{fulllineitems}

\index{EditorGraphicsView (class in gui.DiagramEditor)}

\begin{fulllineitems}
\phantomsection\label{gui_link:gui.DiagramEditor.EditorGraphicsView}\pysiglinewithargsret{\strong{class }\code{gui.DiagramEditor.}\bfcode{EditorGraphicsView}}{\emph{parent=None}}{}
Bases: \code{PyQt4.QtGui.QGraphicsView}

Viewport for the \titleref{gui.DiagramScene} s.

It controls the zooming feature and drag and drop operations from the item library.
\begin{quote}\begin{description}
\item[{Member parent}] \leavevmode
Parent editor widget.

\item[{Member scaleFactor}] \leavevmode
Zooming factor.

\end{description}\end{quote}
\index{dragEnterEvent() (gui.DiagramEditor.EditorGraphicsView method)}

\begin{fulllineitems}
\phantomsection\label{gui_link:gui.DiagramEditor.EditorGraphicsView.dragEnterEvent}\pysiglinewithargsret{\bfcode{dragEnterEvent}}{\emph{event}}{}
Callback method, when an icon is dragged from the \titleref{libraryModelView} to \titleref{gui.diagramScene}.
\begin{quote}\begin{description}
\item[{Parameters}] \leavevmode
\textbf{\texttt{event}} -- \titleref{QtGui.dragEnterEvent}.

\end{description}\end{quote}

\end{fulllineitems}

\index{dragMoveEvent() (gui.DiagramEditor.EditorGraphicsView method)}

\begin{fulllineitems}
\phantomsection\label{gui_link:gui.DiagramEditor.EditorGraphicsView.dragMoveEvent}\pysiglinewithargsret{\bfcode{dragMoveEvent}}{\emph{event}}{}
Callback method, when an icon is moved over the \titleref{gui.diagramScene}.
\begin{quote}\begin{description}
\item[{Parameters}] \leavevmode
\textbf{\texttt{event}} -- \titleref{QtGui.dragMoveEvent}.

\end{description}\end{quote}

\end{fulllineitems}

\index{dropEvent() (gui.DiagramEditor.EditorGraphicsView method)}

\begin{fulllineitems}
\phantomsection\label{gui_link:gui.DiagramEditor.EditorGraphicsView.dropEvent}\pysiglinewithargsret{\bfcode{dropEvent}}{\emph{event}}{}
Callback method, when an icon is dropped on the \titleref{gui.diagramScene}.
\begin{quote}\begin{description}
\item[{Parameters}] \leavevmode
\textbf{\texttt{event}} -- \titleref{QtGui.dropEvent}.

\end{description}\end{quote}

\end{fulllineitems}

\index{validDrop() (gui.DiagramEditor.EditorGraphicsView method)}

\begin{fulllineitems}
\phantomsection\label{gui_link:gui.DiagramEditor.EditorGraphicsView.validDrop}\pysiglinewithargsret{\bfcode{validDrop}}{}{}
Callback method, when the drop on \titleref{gui.DiagramScene} was valid.

\end{fulllineitems}

\index{wheelEvent() (gui.DiagramEditor.EditorGraphicsView method)}

\begin{fulllineitems}
\phantomsection\label{gui_link:gui.DiagramEditor.EditorGraphicsView.wheelEvent}\pysiglinewithargsret{\bfcode{wheelEvent}}{\emph{event}}{}
Callback method, when the mouse wheel is used.
\begin{quote}\begin{description}
\item[{Parameters}] \leavevmode
\textbf{\texttt{event}} -- \titleref{QtGui.wheelEvent}.

\end{description}\end{quote}

\end{fulllineitems}


\end{fulllineitems}

\index{bin\_() (in module gui.DiagramEditor)}

\begin{fulllineitems}
\phantomsection\label{gui_link:gui.DiagramEditor.bin_}\pysiglinewithargsret{\code{gui.DiagramEditor.}\bfcode{bin\_}}{\emph{QTextStream}}{{ $\rightarrow$ QTextStream}}
\end{fulllineitems}

\index{hex\_() (in module gui.DiagramEditor)}

\begin{fulllineitems}
\phantomsection\label{gui_link:gui.DiagramEditor.hex_}\pysiglinewithargsret{\code{gui.DiagramEditor.}\bfcode{hex\_}}{\emph{QTextStream}}{{ $\rightarrow$ QTextStream}}
\end{fulllineitems}

\index{oct\_() (in module gui.DiagramEditor)}

\begin{fulllineitems}
\phantomsection\label{gui_link:gui.DiagramEditor.oct_}\pysiglinewithargsret{\code{gui.DiagramEditor.}\bfcode{oct\_}}{\emph{QTextStream}}{{ $\rightarrow$ QTextStream}}
\end{fulllineitems}

\phantomsection\label{gui_link:module-gui.DiagramScene}\index{gui.DiagramScene (module)}\index{DiagramScene (class in gui.DiagramScene)}

\begin{fulllineitems}
\phantomsection\label{gui_link:gui.DiagramScene.DiagramScene}\pysiglinewithargsret{\strong{class }\code{gui.DiagramScene.}\bfcode{DiagramScene}}{\emph{parent=None}}{}
Bases: \code{PyQt4.QtGui.QGraphicsScene}

Drawing Area.
\begin{quote}\begin{description}
\item[{Member editor}] \leavevmode
Parent \titleref{gui.DiagramEditor}.

\item[{Member hovering}] \leavevmode
Flag determining, whether hovering is happening. (Workaround, since hovering is not forwarded).

\end{description}\end{quote}
\index{mouseMoveEvent() (gui.DiagramScene.DiagramScene method)}

\begin{fulllineitems}
\phantomsection\label{gui_link:gui.DiagramScene.DiagramScene.mouseMoveEvent}\pysiglinewithargsret{\bfcode{mouseMoveEvent}}{\emph{event}}{}
Forward mouseMoveEvent during arc creation and reimplement hovering of \titleref{model.AbstractItem.Connector}.
\begin{quote}\begin{description}
\item[{Parameters}] \leavevmode
\textbf{\texttt{event}} -- \titleref{QtGui.mouseMoveEvent}.

\end{description}\end{quote}

\end{fulllineitems}

\index{mouseReleaseEvent() (gui.DiagramScene.DiagramScene method)}

\begin{fulllineitems}
\phantomsection\label{gui_link:gui.DiagramScene.DiagramScene.mouseReleaseEvent}\pysiglinewithargsret{\bfcode{mouseReleaseEvent}}{\emph{event}}{}
Forward mouseReleaseEvent during arc creation.
\begin{quote}\begin{description}
\item[{Parameters}] \leavevmode
\textbf{\texttt{event}} -- \titleref{QtGui.mouseReleaseEvent}.

\end{description}\end{quote}

\end{fulllineitems}


\end{fulllineitems}

\index{bin\_() (in module gui.DiagramScene)}

\begin{fulllineitems}
\phantomsection\label{gui_link:gui.DiagramScene.bin_}\pysiglinewithargsret{\code{gui.DiagramScene.}\bfcode{bin\_}}{\emph{QTextStream}}{{ $\rightarrow$ QTextStream}}
\end{fulllineitems}

\index{hex\_() (in module gui.DiagramScene)}

\begin{fulllineitems}
\phantomsection\label{gui_link:gui.DiagramScene.hex_}\pysiglinewithargsret{\code{gui.DiagramScene.}\bfcode{hex\_}}{\emph{QTextStream}}{{ $\rightarrow$ QTextStream}}
\end{fulllineitems}

\index{oct\_() (in module gui.DiagramScene)}

\begin{fulllineitems}
\phantomsection\label{gui_link:gui.DiagramScene.oct_}\pysiglinewithargsret{\code{gui.DiagramScene.}\bfcode{oct\_}}{\emph{QTextStream}}{{ $\rightarrow$ QTextStream}}
\end{fulllineitems}

\phantomsection\label{gui_link:module-gui.LibraryModel}\index{gui.LibraryModel (module)}\index{LibraryModel (class in gui.LibraryModel)}

\begin{fulllineitems}
\phantomsection\label{gui_link:gui.LibraryModel.LibraryModel}\pysiglinewithargsret{\strong{class }\code{gui.LibraryModel.}\bfcode{LibraryModel}}{\emph{parent=None}}{}
Bases: \code{PyQt4.QtGui.QStandardItemModel}

Abstract class for drag and drop support
\index{mimeData() (gui.LibraryModel.LibraryModel method)}

\begin{fulllineitems}
\phantomsection\label{gui_link:gui.LibraryModel.LibraryModel.mimeData}\pysiglinewithargsret{\bfcode{mimeData}}{\emph{idxs}}{}
\end{fulllineitems}

\index{mimeTypes() (gui.LibraryModel.LibraryModel method)}

\begin{fulllineitems}
\phantomsection\label{gui_link:gui.LibraryModel.LibraryModel.mimeTypes}\pysiglinewithargsret{\bfcode{mimeTypes}}{}{}
\end{fulllineitems}


\end{fulllineitems}

\index{bin\_() (in module gui.LibraryModel)}

\begin{fulllineitems}
\phantomsection\label{gui_link:gui.LibraryModel.bin_}\pysiglinewithargsret{\code{gui.LibraryModel.}\bfcode{bin\_}}{\emph{QTextStream}}{{ $\rightarrow$ QTextStream}}
\end{fulllineitems}

\index{hex\_() (in module gui.LibraryModel)}

\begin{fulllineitems}
\phantomsection\label{gui_link:gui.LibraryModel.hex_}\pysiglinewithargsret{\code{gui.LibraryModel.}\bfcode{hex\_}}{\emph{QTextStream}}{{ $\rightarrow$ QTextStream}}
\end{fulllineitems}

\index{oct\_() (in module gui.LibraryModel)}

\begin{fulllineitems}
\phantomsection\label{gui_link:gui.LibraryModel.oct_}\pysiglinewithargsret{\code{gui.LibraryModel.}\bfcode{oct\_}}{\emph{QTextStream}}{{ $\rightarrow$ QTextStream}}
\end{fulllineitems}

\phantomsection\label{gui_link:module-gui.NameDialog}\index{gui.NameDialog (module)}\index{NameDialog (class in gui.NameDialog)}

\begin{fulllineitems}
\phantomsection\label{gui_link:gui.NameDialog.NameDialog}\pysiglinewithargsret{\strong{class }\code{gui.NameDialog.}\bfcode{NameDialog}}{\emph{parent=None}, \emph{item=None}, \emph{title='Unnamed'}, \emph{default='a'}}{}
Bases: \code{PyQt4.QtGui.QDialog}
\index{cancel() (gui.NameDialog.NameDialog method)}

\begin{fulllineitems}
\phantomsection\label{gui_link:gui.NameDialog.NameDialog.cancel}\pysiglinewithargsret{\bfcode{cancel}}{}{}
\end{fulllineitems}

\index{getItem() (gui.NameDialog.NameDialog method)}

\begin{fulllineitems}
\phantomsection\label{gui_link:gui.NameDialog.NameDialog.getItem}\pysiglinewithargsret{\bfcode{getItem}}{}{}~\begin{quote}\begin{description}
\item[{Return self.item}] \leavevmode
\end{description}\end{quote}

\end{fulllineitems}

\index{getName() (gui.NameDialog.NameDialog method)}

\begin{fulllineitems}
\phantomsection\label{gui_link:gui.NameDialog.NameDialog.getName}\pysiglinewithargsret{\bfcode{getName}}{}{}~\begin{quote}\begin{description}
\item[{Return self.lineEdit.text()}] \leavevmode
\end{description}\end{quote}

\end{fulllineitems}

\index{ok() (gui.NameDialog.NameDialog method)}

\begin{fulllineitems}
\phantomsection\label{gui_link:gui.NameDialog.NameDialog.ok}\pysiglinewithargsret{\bfcode{ok}}{}{}
\end{fulllineitems}


\end{fulllineitems}

\index{bin\_() (in module gui.NameDialog)}

\begin{fulllineitems}
\phantomsection\label{gui_link:gui.NameDialog.bin_}\pysiglinewithargsret{\code{gui.NameDialog.}\bfcode{bin\_}}{\emph{QTextStream}}{{ $\rightarrow$ QTextStream}}
\end{fulllineitems}

\index{hex\_() (in module gui.NameDialog)}

\begin{fulllineitems}
\phantomsection\label{gui_link:gui.NameDialog.hex_}\pysiglinewithargsret{\code{gui.NameDialog.}\bfcode{hex\_}}{\emph{QTextStream}}{{ $\rightarrow$ QTextStream}}
\end{fulllineitems}

\index{oct\_() (in module gui.NameDialog)}

\begin{fulllineitems}
\phantomsection\label{gui_link:gui.NameDialog.oct_}\pysiglinewithargsret{\code{gui.NameDialog.}\bfcode{oct\_}}{\emph{QTextStream}}{{ $\rightarrow$ QTextStream}}
\end{fulllineitems}

\phantomsection\label{gui_link:module-gui.ParameterDialog}\index{gui.ParameterDialog (module)}\index{ParameterDialog (class in gui.ParameterDialog)}

\begin{fulllineitems}
\phantomsection\label{gui_link:gui.ParameterDialog.ParameterDialog}\pysiglinewithargsret{\strong{class }\code{gui.ParameterDialog.}\bfcode{ParameterDialog}}{\emph{node}, \emph{parent=None}}{}
Bases: \code{PyQt4.QtGui.QDialog}
\index{OK() (gui.ParameterDialog.ParameterDialog method)}

\begin{fulllineitems}
\phantomsection\label{gui_link:gui.ParameterDialog.ParameterDialog.OK}\pysiglinewithargsret{\bfcode{OK}}{}{}
\end{fulllineitems}


\end{fulllineitems}

\index{bin\_() (in module gui.ParameterDialog)}

\begin{fulllineitems}
\phantomsection\label{gui_link:gui.ParameterDialog.bin_}\pysiglinewithargsret{\code{gui.ParameterDialog.}\bfcode{bin\_}}{\emph{QTextStream}}{{ $\rightarrow$ QTextStream}}
\end{fulllineitems}

\index{hex\_() (in module gui.ParameterDialog)}

\begin{fulllineitems}
\phantomsection\label{gui_link:gui.ParameterDialog.hex_}\pysiglinewithargsret{\code{gui.ParameterDialog.}\bfcode{hex\_}}{\emph{QTextStream}}{{ $\rightarrow$ QTextStream}}
\end{fulllineitems}

\index{oct\_() (in module gui.ParameterDialog)}

\begin{fulllineitems}
\phantomsection\label{gui_link:gui.ParameterDialog.oct_}\pysiglinewithargsret{\code{gui.ParameterDialog.}\bfcode{oct\_}}{\emph{QTextStream}}{{ $\rightarrow$ QTextStream}}
\end{fulllineitems}

\phantomsection\label{gui_link:module-gui.SubnetDialog}\index{gui.SubnetDialog (module)}\index{SubnetDialog (class in gui.SubnetDialog)}

\begin{fulllineitems}
\phantomsection\label{gui_link:gui.SubnetDialog.SubnetDialog}\pysiglinewithargsret{\strong{class }\code{gui.SubnetDialog.}\bfcode{SubnetDialog}}{\emph{mainWindow}, \emph{parent=None}, \emph{superNet=None}, \emph{subnet=None}}{}
Bases: \code{PyQt4.QtGui.QDialog}

Dialog for subnet creation
\index{cancel() (gui.SubnetDialog.SubnetDialog method)}

\begin{fulllineitems}
\phantomsection\label{gui_link:gui.SubnetDialog.SubnetDialog.cancel}\pysiglinewithargsret{\bfcode{cancel}}{}{}
\end{fulllineitems}

\index{closeEvent() (gui.SubnetDialog.SubnetDialog method)}

\begin{fulllineitems}
\phantomsection\label{gui_link:gui.SubnetDialog.SubnetDialog.closeEvent}\pysiglinewithargsret{\bfcode{closeEvent}}{\emph{event}}{}~\begin{quote}\begin{description}
\item[{Parameters}] \leavevmode
\textbf{\texttt{event}} -- 

\end{description}\end{quote}

\end{fulllineitems}


\end{fulllineitems}

\index{bin\_() (in module gui.SubnetDialog)}

\begin{fulllineitems}
\phantomsection\label{gui_link:gui.SubnetDialog.bin_}\pysiglinewithargsret{\code{gui.SubnetDialog.}\bfcode{bin\_}}{\emph{QTextStream}}{{ $\rightarrow$ QTextStream}}
\end{fulllineitems}

\index{hex\_() (in module gui.SubnetDialog)}

\begin{fulllineitems}
\phantomsection\label{gui_link:gui.SubnetDialog.hex_}\pysiglinewithargsret{\code{gui.SubnetDialog.}\bfcode{hex\_}}{\emph{QTextStream}}{{ $\rightarrow$ QTextStream}}
\end{fulllineitems}

\index{oct\_() (in module gui.SubnetDialog)}

\begin{fulllineitems}
\phantomsection\label{gui_link:gui.SubnetDialog.oct_}\pysiglinewithargsret{\code{gui.SubnetDialog.}\bfcode{oct\_}}{\emph{QTextStream}}{{ $\rightarrow$ QTextStream}}
\end{fulllineitems}

\phantomsection\label{gui_link:module-gui.TokenDialog}\index{gui.TokenDialog (module)}\index{TokenDialog (class in gui.TokenDialog)}

\begin{fulllineitems}
\phantomsection\label{gui_link:gui.TokenDialog.TokenDialog}\pysiglinewithargsret{\strong{class }\code{gui.TokenDialog.}\bfcode{TokenDialog}}{\emph{parent=None}, \emph{title='Choose Colour and Amount of the Token'}}{}
Bases: \code{PyQt4.QtGui.QDialog}

Dialog to choose a color for new token
\index{cancel() (gui.TokenDialog.TokenDialog method)}

\begin{fulllineitems}
\phantomsection\label{gui_link:gui.TokenDialog.TokenDialog.cancel}\pysiglinewithargsret{\bfcode{cancel}}{}{}
\end{fulllineitems}

\index{getCountToken() (gui.TokenDialog.TokenDialog method)}

\begin{fulllineitems}
\phantomsection\label{gui_link:gui.TokenDialog.TokenDialog.getCountToken}\pysiglinewithargsret{\bfcode{getCountToken}}{}{}~\begin{quote}\begin{description}
\item[{Return self.countToken}] \leavevmode
\end{description}\end{quote}

\end{fulllineitems}

\index{getInitMarking() (gui.TokenDialog.TokenDialog method)}

\begin{fulllineitems}
\phantomsection\label{gui_link:gui.TokenDialog.TokenDialog.getInitMarking}\pysiglinewithargsret{\bfcode{getInitMarking}}{}{}~\begin{quote}\begin{description}
\item[{Return self.initMarking}] \leavevmode
\end{description}\end{quote}

\end{fulllineitems}

\index{getListEntry() (gui.TokenDialog.TokenDialog method)}

\begin{fulllineitems}
\phantomsection\label{gui_link:gui.TokenDialog.TokenDialog.getListEntry}\pysiglinewithargsret{\bfcode{getListEntry}}{}{}~\begin{quote}\begin{description}
\item[{Return self.listEntry}] \leavevmode
\end{description}\end{quote}

\end{fulllineitems}

\index{ok() (gui.TokenDialog.TokenDialog method)}

\begin{fulllineitems}
\phantomsection\label{gui_link:gui.TokenDialog.TokenDialog.ok}\pysiglinewithargsret{\bfcode{ok}}{}{}
\end{fulllineitems}

\index{setCountToken() (gui.TokenDialog.TokenDialog method)}

\begin{fulllineitems}
\phantomsection\label{gui_link:gui.TokenDialog.TokenDialog.setCountToken}\pysiglinewithargsret{\bfcode{setCountToken}}{\emph{value}}{}
Set token count.
\begin{quote}\begin{description}
\item[{Parameters}] \leavevmode
\textbf{\texttt{value}} -- 

\end{description}\end{quote}

\end{fulllineitems}

\index{setInitMarking() (gui.TokenDialog.TokenDialog method)}

\begin{fulllineitems}
\phantomsection\label{gui_link:gui.TokenDialog.TokenDialog.setInitMarking}\pysiglinewithargsret{\bfcode{setInitMarking}}{\emph{value}}{}
Set initial Marking.
\begin{quote}\begin{description}
\item[{Parameters}] \leavevmode
\textbf{\texttt{value}} -- 

\end{description}\end{quote}

\end{fulllineitems}

\index{setListEntry() (gui.TokenDialog.TokenDialog method)}

\begin{fulllineitems}
\phantomsection\label{gui_link:gui.TokenDialog.TokenDialog.setListEntry}\pysiglinewithargsret{\bfcode{setListEntry}}{}{}
Add new color to list.

\end{fulllineitems}


\end{fulllineitems}

\index{bin\_() (in module gui.TokenDialog)}

\begin{fulllineitems}
\phantomsection\label{gui_link:gui.TokenDialog.bin_}\pysiglinewithargsret{\code{gui.TokenDialog.}\bfcode{bin\_}}{\emph{QTextStream}}{{ $\rightarrow$ QTextStream}}
\end{fulllineitems}

\index{hex\_() (in module gui.TokenDialog)}

\begin{fulllineitems}
\phantomsection\label{gui_link:gui.TokenDialog.hex_}\pysiglinewithargsret{\code{gui.TokenDialog.}\bfcode{hex\_}}{\emph{QTextStream}}{{ $\rightarrow$ QTextStream}}
\end{fulllineitems}

\index{oct\_() (in module gui.TokenDialog)}

\begin{fulllineitems}
\phantomsection\label{gui_link:gui.TokenDialog.oct_}\pysiglinewithargsret{\code{gui.TokenDialog.}\bfcode{oct\_}}{\emph{QTextStream}}{{ $\rightarrow$ QTextStream}}
\end{fulllineitems}



\chapter{model}
\label{model_link:module-model.CPNSimulator}\label{model_link:model}\label{model_link::doc}\index{model.CPNSimulator (module)}\index{CPNSimulator (class in model.CPNSimulator)}

\begin{fulllineitems}
\phantomsection\label{model_link:model.CPNSimulator.CPNSimulator}\pysiglinewithargsret{\strong{class }\code{model.CPNSimulator.}\bfcode{CPNSimulator}}{\emph{mainWindow}}{}
Bases: \code{object}

CPN Simulator.
\begin{quote}\begin{description}
\item[{Member mainWindow}] \leavevmode
\titleref{gui.MainWindow}. Main application window.

\item[{Member net}] \leavevmode
SNAKES Colored Petrinet.

\item[{Member markingHistory}] \leavevmode
SNAKES Marking stored for every {}` simulationStep{}`.

\item[{Member simulationStep}] \leavevmode
Steps calculated for the Petrinet \titleref{net}

\item[{Member displayStep}] \leavevmode
Step displayed in editors.

\item[{Member simulatorSpeed}] \leavevmode
The speed with which the simulator progresses.

\item[{Member enabledTransitions}] \leavevmode
Number of enabled transitions.

\item[{Member uniqueNameBase}] \leavevmode
Unique integer for the creation of new CPN elements.

\item[{Member initialMarking}] \leavevmode
SNAKES Marking at step 0.

\item[{Member colourSets}] \leavevmode
Set of defined string colors.

\item[{Member connectionList}] \leavevmode
List of all \titleref{model.ArcItem} s, except port substitution transition connections.

\item[{Member transitions}] \leavevmode
All \titleref{model.TransitionItem} except substitution transitions.

\item[{Member places}] \leavevmode
All \titleref{model.PlaceItem} except port places.

\item[{Member subnets}] \leavevmode
List of all subnets.

\end{description}\end{quote}
\index{\_\_init\_\_() (model.CPNSimulator.CPNSimulator method)}

\begin{fulllineitems}
\phantomsection\label{model_link:model.CPNSimulator.CPNSimulator.__init__}\pysiglinewithargsret{\bfcode{\_\_init\_\_}}{\emph{mainWindow}}{}
Create CPN Simulator.
\begin{quote}\begin{description}
\item[{Parameters}] \leavevmode
\textbf{\texttt{mainWindow}} -- \titleref{gui.MainWindow}. Main application window.

\end{description}\end{quote}

\end{fulllineitems}

\index{back2beginning() (model.CPNSimulator.CPNSimulator method)}

\begin{fulllineitems}
\phantomsection\label{model_link:model.CPNSimulator.CPNSimulator.back2beginning}\pysiglinewithargsret{\bfcode{back2beginning}}{}{}
Return to step 0.

\end{fulllineitems}

\index{backStep() (model.CPNSimulator.CPNSimulator method)}

\begin{fulllineitems}
\phantomsection\label{model_link:model.CPNSimulator.CPNSimulator.backStep}\pysiglinewithargsret{\bfcode{backStep}}{}{}
Go one step back in history.

\end{fulllineitems}

\index{checkTranistionActivation() (model.CPNSimulator.CPNSimulator method)}

\begin{fulllineitems}
\phantomsection\label{model_link:model.CPNSimulator.CPNSimulator.checkTranistionActivation}\pysiglinewithargsret{\bfcode{checkTranistionActivation}}{\emph{transition}, \emph{mode}, \emph{currentStep}}{}
Check whether a SNAKES transition is activated.
\begin{quote}\begin{description}
\item[{Parameters}] \leavevmode\begin{itemize}
\item {} 
\textbf{\texttt{transition}} -- SNAKES transition to check.

\item {} 
\textbf{\texttt{mode}} -- SNAKES mode to check for activation.

\item {} 
\textbf{\texttt{currentStep}} -- Step for which to calculate the activation

\end{itemize}

\item[{Return activated}] \leavevmode
Activated True or False.

\end{description}\end{quote}

\end{fulllineitems}

\index{checkTransitionEnabled() (model.CPNSimulator.CPNSimulator method)}

\begin{fulllineitems}
\phantomsection\label{model_link:model.CPNSimulator.CPNSimulator.checkTransitionEnabled}\pysiglinewithargsret{\bfcode{checkTransitionEnabled}}{\emph{transition}, \emph{mode}, \emph{currentStep}, \emph{activated}, \emph{transitions2Fire}}{}
Check whether a SNAKES transition is enabled.
\begin{quote}\begin{description}
\item[{Parameters}] \leavevmode\begin{itemize}
\item {} 
\textbf{\texttt{transition}} -- SNAKES transition to check.

\item {} 
\textbf{\texttt{mode}} -- SNAKES mode to check for activation.

\item {} 
\textbf{\texttt{currentStep}} -- Step for which to calculate the enabling.

\item {} 
\textbf{\texttt{activated}} -- Flag that determines, whether \titleref{transition} is activated.

\item {} 
\textbf{\texttt{transitions2Fire}} -- List of SNAKES transitions enabled to fire.

\end{itemize}

\item[{Return enabled}] \leavevmode
Enabled True or False.

\end{description}\end{quote}

\end{fulllineitems}

\index{defineNewColour() (model.CPNSimulator.CPNSimulator method)}

\begin{fulllineitems}
\phantomsection\label{model_link:model.CPNSimulator.CPNSimulator.defineNewColour}\pysiglinewithargsret{\bfcode{defineNewColour}}{\emph{colName}}{}
Define a new Color.
\begin{quote}\begin{description}
\item[{Parameters}] \leavevmode
\textbf{\texttt{colName}} -- String color.

\end{description}\end{quote}

\end{fulllineitems}

\index{fireEnabledTransitions() (model.CPNSimulator.CPNSimulator method)}

\begin{fulllineitems}
\phantomsection\label{model_link:model.CPNSimulator.CPNSimulator.fireEnabledTransitions}\pysiglinewithargsret{\bfcode{fireEnabledTransitions}}{\emph{currentStep}}{}
Fire transitions in list \titleref{transitions2Fire}.
\begin{quote}\begin{description}
\item[{Parameters}] \leavevmode
\textbf{\texttt{currentStep}} -- Step for which to calculate the firing.

\end{description}\end{quote}

\end{fulllineitems}

\index{forward2lastStep() (model.CPNSimulator.CPNSimulator method)}

\begin{fulllineitems}
\phantomsection\label{model_link:model.CPNSimulator.CPNSimulator.forward2lastStep}\pysiglinewithargsret{\bfcode{forward2lastStep}}{}{}
Go forward to last step in history, given at {}` simulationStep{}`.

\end{fulllineitems}

\index{forwardStep() (model.CPNSimulator.CPNSimulator method)}

\begin{fulllineitems}
\phantomsection\label{model_link:model.CPNSimulator.CPNSimulator.forwardStep}\pysiglinewithargsret{\bfcode{forwardStep}}{}{}
Go step forward in history or claculate new step.

\end{fulllineitems}

\index{getActualMarking() (model.CPNSimulator.CPNSimulator method)}

\begin{fulllineitems}
\phantomsection\label{model_link:model.CPNSimulator.CPNSimulator.getActualMarking}\pysiglinewithargsret{\bfcode{getActualMarking}}{\emph{actualMarking}}{}
Process status of visual net, depending on \titleref{actualMarking}.
\begin{quote}\begin{description}
\item[{Parameters}] \leavevmode
\textbf{\texttt{actualMarking}} -- SNAKES Marking.

\end{description}\end{quote}

\end{fulllineitems}

\index{resetSimulator() (model.CPNSimulator.CPNSimulator method)}

\begin{fulllineitems}
\phantomsection\label{model_link:model.CPNSimulator.CPNSimulator.resetSimulator}\pysiglinewithargsret{\bfcode{resetSimulator}}{\emph{init=False}}{}
Reset simulator history to step 0.
\begin{quote}\begin{description}
\item[{Parameters}] \leavevmode
\textbf{\texttt{init}} -- Initialization on startup/load

\end{description}\end{quote}

\end{fulllineitems}

\index{setNetName() (model.CPNSimulator.CPNSimulator method)}

\begin{fulllineitems}
\phantomsection\label{model_link:model.CPNSimulator.CPNSimulator.setNetName}\pysiglinewithargsret{\bfcode{setNetName}}{}{}
Set the name of the \titleref{net}.

\end{fulllineitems}

\index{startSim() (model.CPNSimulator.CPNSimulator method)}

\begin{fulllineitems}
\phantomsection\label{model_link:model.CPNSimulator.CPNSimulator.startSim}\pysiglinewithargsret{\bfcode{startSim}}{}{}
Start simulator with the speed chosen with the radio edit.

\end{fulllineitems}

\index{stopSim() (model.CPNSimulator.CPNSimulator method)}

\begin{fulllineitems}
\phantomsection\label{model_link:model.CPNSimulator.CPNSimulator.stopSim}\pysiglinewithargsret{\bfcode{stopSim}}{}{}
Stop simulator.

\end{fulllineitems}

\index{tokenAdded() (model.CPNSimulator.CPNSimulator method)}

\begin{fulllineitems}
\phantomsection\label{model_link:model.CPNSimulator.CPNSimulator.tokenAdded}\pysiglinewithargsret{\bfcode{tokenAdded}}{}{}
Called when a \titleref{model.TokenItem} was added to the \titleref{net}.

\end{fulllineitems}


\end{fulllineitems}

\index{bin\_() (in module model.CPNSimulator)}

\begin{fulllineitems}
\phantomsection\label{model_link:model.CPNSimulator.bin_}\pysiglinewithargsret{\code{model.CPNSimulator.}\bfcode{bin\_}}{\emph{QTextStream}}{{ $\rightarrow$ QTextStream}}
\end{fulllineitems}

\index{hex\_() (in module model.CPNSimulator)}

\begin{fulllineitems}
\phantomsection\label{model_link:model.CPNSimulator.hex_}\pysiglinewithargsret{\code{model.CPNSimulator.}\bfcode{hex\_}}{\emph{QTextStream}}{{ $\rightarrow$ QTextStream}}
\end{fulllineitems}

\index{oct\_() (in module model.CPNSimulator)}

\begin{fulllineitems}
\phantomsection\label{model_link:model.CPNSimulator.oct_}\pysiglinewithargsret{\code{model.CPNSimulator.}\bfcode{oct\_}}{\emph{QTextStream}}{{ $\rightarrow$ QTextStream}}
\end{fulllineitems}

\phantomsection\label{model_link:module-model.AbstractItem}\index{model.AbstractItem (module)}\index{AbstractItem (class in model.AbstractItem)}

\begin{fulllineitems}
\phantomsection\label{model_link:model.AbstractItem.AbstractItem}\pysiglinewithargsret{\strong{class }\code{model.AbstractItem.}\bfcode{AbstractItem}}{\emph{parent=None}}{}
Bases: \code{PyQt4.QtGui.QGraphicsItem}

Base class of the CPN elements transition and place.
\begin{quote}\begin{description}
\item[{Member parent}] \leavevmode
\titleref{gui.DiagramEditor}. Editor to show in.

\item[{Member tokens}] \leavevmode
List of \titleref{model.TokenItem} s, if inheriting class is \titleref{model.PlaceItem} else None

\item[{Member name}] \leavevmode
Name of the CPN element.

\item[{Member posCallbacks}] \leavevmode
List of callback functions, to calculate position changes.

\item[{Member connectorList}] \leavevmode
List of 20 \titleref{Connector} s

\item[{Member planeMap}] \leavevmode
Dictionary-\textgreater{}Set(), mapping the relative orientation of an any other \titleref{model.AbstractItem} in the containing editor.

\item[{Member connectorMap}] \leavevmode
Dictionary-\textgreater{}List(), mapping the relative orientation of the 20 \titleref{model.AbstractItem.Connector} s.

\item[{Member nodeType}] \leavevmode
Determining the type of the item at creation time.

\item[{Member superNet}] \leavevmode
The super net is used to determine the editor for port place clones.

\item[{Member label}] \leavevmode
\titleref{QtGui.QGraphicsTextItem}, visual representation of the \titleref{model.AbstractItem} s name.

\item[{Member descCanvas}] \leavevmode
\titleref{gui.AbstractItem.DescriptionCanvas}, showing detailed information.

\end{description}\end{quote}
\index{\_\_init\_\_() (model.AbstractItem.AbstractItem method)}

\begin{fulllineitems}
\phantomsection\label{model_link:model.AbstractItem.AbstractItem.__init__}\pysiglinewithargsret{\bfcode{\_\_init\_\_}}{\emph{parent=None}}{}
Create abstract item.
\begin{quote}\begin{description}
\item[{Parameters}] \leavevmode
\textbf{\texttt{parent}} -- \titleref{gui.DiagramEditor}. Editor to show in.

\end{description}\end{quote}

\end{fulllineitems}

\index{checkItem() (model.AbstractItem.AbstractItem method)}

\begin{fulllineitems}
\phantomsection\label{model_link:model.AbstractItem.AbstractItem.checkItem}\pysiglinewithargsret{\bfcode{checkItem}}{\emph{item}, \emph{orientation}}{}
Check \titleref{item} is of propper type.
\begin{quote}\begin{description}
\item[{Parameters}] \leavevmode\begin{itemize}
\item {} 
\textbf{\texttt{item}} -- \titleref{model.AbstractItem} to lookup.

\item {} 
\textbf{\texttt{orientation}} -- Orientation to assign, if type check is passed.

\end{itemize}

\end{description}\end{quote}

\end{fulllineitems}

\index{createItem() (model.AbstractItem.AbstractItem method)}

\begin{fulllineitems}
\phantomsection\label{model_link:model.AbstractItem.AbstractItem.createItem}\pysiglinewithargsret{\bfcode{createItem}}{\emph{editor}, \emph{name='Untitled'}, \emph{nodeType='undefined'}}{}
Create the typed item.
\begin{quote}\begin{description}
\item[{Parameters}] \leavevmode\begin{itemize}
\item {} 
\textbf{\texttt{editor}} -- \titleref{gui.DiagramEditor}. Editor to show in.

\item {} 
\textbf{\texttt{name}} -- Name of the CPN element.

\item {} 
\textbf{\texttt{nodeType}} -- Determining the type of the item at creation time.

\end{itemize}

\item[{Return w, h}] \leavevmode
Width and height of CPN element, depending on the label length.

\end{description}\end{quote}

\end{fulllineitems}

\index{deleteItemLocal() (model.AbstractItem.AbstractItem method)}

\begin{fulllineitems}
\phantomsection\label{model_link:model.AbstractItem.AbstractItem.deleteItemLocal}\pysiglinewithargsret{\bfcode{deleteItemLocal}}{}{}
Capture delete event and call editor delete function.

\end{fulllineitems}

\index{findItemsInPlanes() (model.AbstractItem.AbstractItem method)}

\begin{fulllineitems}
\phantomsection\label{model_link:model.AbstractItem.AbstractItem.findItemsInPlanes}\pysiglinewithargsret{\bfcode{findItemsInPlanes}}{}{}
Finds the orientation of any other item in the \titleref{gui.DiagramScene}.

\end{fulllineitems}

\index{mouseDoubleClickEvent() (model.AbstractItem.AbstractItem method)}

\begin{fulllineitems}
\phantomsection\label{model_link:model.AbstractItem.AbstractItem.mouseDoubleClickEvent}\pysiglinewithargsret{\bfcode{mouseDoubleClickEvent}}{\emph{event}}{}
Edit visual name on \titleref{QtGui.mouseDoubleClickEvent}.
\begin{quote}\begin{description}
\item[{Parameters}] \leavevmode
\textbf{\texttt{event}} -- \titleref{QtGui.mouseDoubleClickEvent}.

\end{description}\end{quote}

\end{fulllineitems}

\index{renameElement() (model.AbstractItem.AbstractItem method)}

\begin{fulllineitems}
\phantomsection\label{model_link:model.AbstractItem.AbstractItem.renameElement}\pysiglinewithargsret{\bfcode{renameElement}}{}{}
Capture rename event and call virtual function of \titleref{model.PlaceItem} or \titleref{model.TrasnitionItem}.

\end{fulllineitems}


\end{fulllineitems}

\index{Connector (class in model.AbstractItem)}

\begin{fulllineitems}
\phantomsection\label{model_link:model.AbstractItem.Connector}\pysiglinewithargsret{\strong{class }\code{model.AbstractItem.}\bfcode{Connector}}{\emph{parent}, \emph{idx}}{}
Bases: \code{PyQt4.QtGui.QGraphicsRectItem}

Connector for visual arcs.

A connector is a socket, to or from which a visual arc may be connected.
\begin{quote}\begin{description}
\item[{Member parent}] \leavevmode
Parent port place.

\item[{Member orientation}] \leavevmode
Orientation relative to its parent: ``N'',''NE'',''E'',''SE'',''S'',''SW'',''W'',''NW''.

\item[{Member idx}] \leavevmode
Index assinged from parent.

\item[{Member connectionArc}] \leavevmode
Reference to the arc connected to this Connector.

\item[{Member posCallbacks}] \leavevmode
List of callback functions, to calculate position changes.

\item[{Member position}] \leavevmode
Position relative to parent.

\end{description}\end{quote}
\index{\_\_init\_\_() (model.AbstractItem.Connector method)}

\begin{fulllineitems}
\phantomsection\label{model_link:model.AbstractItem.Connector.__init__}\pysiglinewithargsret{\bfcode{\_\_init\_\_}}{\emph{parent}, \emph{idx}}{}
Create a connector.
\begin{quote}\begin{description}
\item[{Parameters}] \leavevmode\begin{itemize}
\item {} 
\textbf{\texttt{parent}} -- Parent AbstractItem.

\item {} 
\textbf{\texttt{idx}} -- Absolut number in the parent list of connectors.

\end{itemize}

\end{description}\end{quote}

\end{fulllineitems}

\index{hoverEnterEvent() (model.AbstractItem.Connector method)}

\begin{fulllineitems}
\phantomsection\label{model_link:model.AbstractItem.Connector.hoverEnterEvent}\pysiglinewithargsret{\bfcode{hoverEnterEvent}}{\emph{event}}{}
Make connector visible on hoverEnterEvent.
\begin{quote}\begin{description}
\item[{Parameters}] \leavevmode
\textbf{\texttt{event}} -- hoverEnterEvent

\end{description}\end{quote}

\end{fulllineitems}

\index{hoverLeaveEvent() (model.AbstractItem.Connector method)}

\begin{fulllineitems}
\phantomsection\label{model_link:model.AbstractItem.Connector.hoverLeaveEvent}\pysiglinewithargsret{\bfcode{hoverLeaveEvent}}{\emph{event}}{}
Make connector invisible on hoverLeaveEvent.
\begin{quote}\begin{description}
\item[{Parameters}] \leavevmode
\textbf{\texttt{event}} -- hoverLeaveEvent

\end{description}\end{quote}

\end{fulllineitems}

\index{itemChange() (model.AbstractItem.Connector method)}

\begin{fulllineitems}
\phantomsection\label{model_link:model.AbstractItem.Connector.itemChange}\pysiglinewithargsret{\bfcode{itemChange}}{\emph{change}, \emph{value}}{}
Item position has changed, calculate new position.
\begin{quote}\begin{description}
\item[{Parameters}] \leavevmode\begin{itemize}
\item {} 
\textbf{\texttt{change}} -- Change event.

\item {} 
\textbf{\texttt{value}} -- QtCore.QPointF().

\end{itemize}

\item[{Return value}] \leavevmode
QtCore.QPointF(x, y) or super(Connector, self).itemChange(change, value).

\end{description}\end{quote}

\end{fulllineitems}

\index{mousePressEvent() (model.AbstractItem.Connector method)}

\begin{fulllineitems}
\phantomsection\label{model_link:model.AbstractItem.Connector.mousePressEvent}\pysiglinewithargsret{\bfcode{mousePressEvent}}{\emph{event}}{}
Capture QtGui.mousePressEvent with \textbf{Shift-Key} modifier.
\begin{quote}\begin{description}
\item[{Parameters}] \leavevmode
\textbf{\texttt{event}} -- QtGui.mousePressEvent

\end{description}\end{quote}

\end{fulllineitems}


\end{fulllineitems}

\index{DescriptionCanvas (class in model.AbstractItem)}

\begin{fulllineitems}
\phantomsection\label{model_link:model.AbstractItem.DescriptionCanvas}\pysiglinewithargsret{\strong{class }\code{model.AbstractItem.}\bfcode{DescriptionCanvas}}{\emph{parent=None}}{}
Bases: \code{PyQt4.QtGui.QGraphicsRectItem}

Description area for tokens, exceptions and modes. Toggle with \textbf{Ctrl**+**M}.

In this canvas label, detailed information about the CPN element is shown.
Firstly the unique name used by the simulator is shown, followed by the visual name, 
that is assigned by the user. The latter does not have to be unique.
\begin{quote}\begin{description}
\item[{Member parent}] \leavevmode
Parent CPN element.

\item[{Member label}] \leavevmode
Visual representation of the information about the CPN element.

\item[{Member text}] \leavevmode
Unique name of parent CPN element.

\item[{Member visibility}] \leavevmode
Switch for the visibility of the description canvas.

\end{description}\end{quote}
\index{\_\_init\_\_() (model.AbstractItem.DescriptionCanvas method)}

\begin{fulllineitems}
\phantomsection\label{model_link:model.AbstractItem.DescriptionCanvas.__init__}\pysiglinewithargsret{\bfcode{\_\_init\_\_}}{\emph{parent=None}}{}
Create description canvas.
\begin{quote}\begin{description}
\item[{Parameters}] \leavevmode
\textbf{\texttt{parent}} -- Abstract CPN element.

\end{description}\end{quote}

\end{fulllineitems}

\index{setCanvasString() (model.AbstractItem.DescriptionCanvas method)}

\begin{fulllineitems}
\phantomsection\label{model_link:model.AbstractItem.DescriptionCanvas.setCanvasString}\pysiglinewithargsret{\bfcode{setCanvasString}}{\emph{infoString}}{}
Set info string, that appended to the first line.
\begin{quote}\begin{description}
\item[{Parameters}] \leavevmode
\textbf{\texttt{infoString}} -- Info string, may contain line breaks.

\end{description}\end{quote}

\end{fulllineitems}

\index{setVisibility() (model.AbstractItem.DescriptionCanvas method)}

\begin{fulllineitems}
\phantomsection\label{model_link:model.AbstractItem.DescriptionCanvas.setVisibility}\pysiglinewithargsret{\bfcode{setVisibility}}{\emph{visible}}{}
Set visibility of DescriptionCanvas.
\begin{quote}\begin{description}
\item[{Parameters}] \leavevmode
\textbf{\texttt{visible}} -- True if visible, False if invisible.

\end{description}\end{quote}

\end{fulllineitems}


\end{fulllineitems}

\index{bin\_() (in module model.AbstractItem)}

\begin{fulllineitems}
\phantomsection\label{model_link:model.AbstractItem.bin_}\pysiglinewithargsret{\code{model.AbstractItem.}\bfcode{bin\_}}{\emph{QTextStream}}{{ $\rightarrow$ QTextStream}}
\end{fulllineitems}

\index{hex\_() (in module model.AbstractItem)}

\begin{fulllineitems}
\phantomsection\label{model_link:model.AbstractItem.hex_}\pysiglinewithargsret{\code{model.AbstractItem.}\bfcode{hex\_}}{\emph{QTextStream}}{{ $\rightarrow$ QTextStream}}
\end{fulllineitems}

\index{oct\_() (in module model.AbstractItem)}

\begin{fulllineitems}
\phantomsection\label{model_link:model.AbstractItem.oct_}\pysiglinewithargsret{\code{model.AbstractItem.}\bfcode{oct\_}}{\emph{QTextStream}}{{ $\rightarrow$ QTextStream}}
\end{fulllineitems}

\phantomsection\label{model_link:module-model.PlaceItem}\index{model.PlaceItem (module)}\index{PlaceItem (class in model.PlaceItem)}

\begin{fulllineitems}
\phantomsection\label{model_link:model.PlaceItem.PlaceItem}\pysiglinewithargsret{\strong{class }\code{model.PlaceItem.}\bfcode{PlaceItem}}{\emph{editor}, \emph{name}, \emph{position}, \emph{initMarking={[}{]}}, \emph{uniqueName='p0'}, \emph{port=None}, \emph{loadFromFile=False}, \emph{portDirection=None}, \emph{portClone=None}, \emph{superNet=None}, \emph{subnet=None}}{}
Bases: \code{PyQt4.QtGui.QGraphicsEllipseItem}, {\hyperref[model_link:model.AbstractItem.AbstractItem]{\crossref{\code{model.AbstractItem.AbstractItem}}}}

CPN Transition element.
\begin{quote}\begin{description}
\item[{Member editor}] \leavevmode
\titleref{gui.DiagramEditor}. Editor to show in.

\item[{Member name}] \leavevmode
Name of the CPN element assigned by the user.

\item[{Member subnet}] \leavevmode
Name of the subnet this transition is contained in.

\item[{Member position}] \leavevmode
Position of this item in the \titleref{gui.DiagramScene}.

\item[{Member port}] \leavevmode
\titleref{model.PortItem} if this is a port place, else None.

\item[{Member portClone}] \leavevmode
A port place in the super net or None.

\item[{Member portDirection}] \leavevmode
Direction of the port, if this is a port place.

\item[{Member uniqueName}] \leavevmode
Unique name of this transition used by the simulator.

\item[{Member place}] \leavevmode
SNAKES place node with id \titleref{uniqueName}, except portClones.

\item[{Member initMarking}] \leavevmode
List of initial tokens present in this place.

\item[{Member tokens}] \leavevmode
List ( deque ) of \titleref{model.TokenItem} s in this place.

\item[{Member toolTipString}] \leavevmode
Information string containing the tokens present in this place.

\end{description}\end{quote}
\index{\_\_init\_\_() (model.PlaceItem.PlaceItem method)}

\begin{fulllineitems}
\phantomsection\label{model_link:model.PlaceItem.PlaceItem.__init__}\pysiglinewithargsret{\bfcode{\_\_init\_\_}}{\emph{editor}, \emph{name}, \emph{position}, \emph{initMarking={[}{]}}, \emph{uniqueName='p0'}, \emph{port=None}, \emph{loadFromFile=False}, \emph{portDirection=None}, \emph{portClone=None}, \emph{superNet=None}, \emph{subnet=None}}{}
Create and initialize place item.
\begin{quote}\begin{description}
\item[{Parameters}] \leavevmode\begin{itemize}
\item {} 
\textbf{\texttt{editor}} -- \titleref{gui.DiagramEditor}. Editor to show in.

\item {} 
\textbf{\texttt{name}} -- Name of the CPN element assigned by the user.

\item {} 
\textbf{\texttt{position}} -- Position of this item in the \titleref{gui.DiagramScene}.

\item {} 
\textbf{\texttt{initMarking}} -- List of initial tokens present in this place.

\item {} 
\textbf{\texttt{uniqueName}} -- Unique name of this transition used by the simulator.

\item {} 
\textbf{\texttt{port}} -- \titleref{model.PortItem} if this is a port place, else None.

\item {} 
\textbf{\texttt{loadFromFile}} -- Flag determining, whether the place is loaded from file.

\item {} 
\textbf{\texttt{portDirection}} -- Direction of the port, if this is a port place.

\item {} 
\textbf{\texttt{portClone}} -- A port place in the super net or None.

\item {} 
\textbf{\texttt{superNe}} -- The super net is used to determine the editor for port place clones.

\item {} 
\textbf{\texttt{subnet}} -- Name of the subnet this transition is contained in.

\end{itemize}

\end{description}\end{quote}

\end{fulllineitems}

\index{addToken() (model.PlaceItem.PlaceItem method)}

\begin{fulllineitems}
\phantomsection\label{model_link:model.PlaceItem.PlaceItem.addToken}\pysiglinewithargsret{\bfcode{addToken}}{}{}
Method to add a \titleref{model.TokenItem} to this place

\end{fulllineitems}

\index{contextMenuEvent() (model.PlaceItem.PlaceItem method)}

\begin{fulllineitems}
\phantomsection\label{model_link:model.PlaceItem.PlaceItem.contextMenuEvent}\pysiglinewithargsret{\bfcode{contextMenuEvent}}{\emph{event}}{}
Generate Context menu on context menu event.
\begin{quote}\begin{description}
\item[{Parameters}] \leavevmode
\textbf{\texttt{event}} -- QContextMenuEvent.

\end{description}\end{quote}

\end{fulllineitems}

\index{deleteItemLocal() (model.PlaceItem.PlaceItem method)}

\begin{fulllineitems}
\phantomsection\label{model_link:model.PlaceItem.PlaceItem.deleteItemLocal}\pysiglinewithargsret{\bfcode{deleteItemLocal}}{}{}
Capture delete event and call editor delete function.

\end{fulllineitems}

\index{mouseMoveEvent() (model.PlaceItem.PlaceItem method)}

\begin{fulllineitems}
\phantomsection\label{model_link:model.PlaceItem.PlaceItem.mouseMoveEvent}\pysiglinewithargsret{\bfcode{mouseMoveEvent}}{\emph{event}}{}
Prevents the movement of this item, when connections are drawn.
\begin{quote}\begin{description}
\item[{Parameters}] \leavevmode
\textbf{\texttt{event}} -- \titleref{QtGui.mouseMoveEvent}

\end{description}\end{quote}

\end{fulllineitems}

\index{mousePressEvent() (model.PlaceItem.PlaceItem method)}

\begin{fulllineitems}
\phantomsection\label{model_link:model.PlaceItem.PlaceItem.mousePressEvent}\pysiglinewithargsret{\bfcode{mousePressEvent}}{\emph{event}}{}
Prevents the movement of this item, when connections are drawn.
\begin{quote}\begin{description}
\item[{Parameters}] \leavevmode
\textbf{\texttt{event}} -- \titleref{QtGui.mousePressEvent}

\end{description}\end{quote}

\end{fulllineitems}

\index{mouseReleaseEvent() (model.PlaceItem.PlaceItem method)}

\begin{fulllineitems}
\phantomsection\label{model_link:model.PlaceItem.PlaceItem.mouseReleaseEvent}\pysiglinewithargsret{\bfcode{mouseReleaseEvent}}{\emph{event}}{}
Forward \titleref{QtGui.mouseReleaseEvent}.
\begin{quote}\begin{description}
\item[{Parameters}] \leavevmode
\textbf{\texttt{event}} -- \titleref{QtGui.mouseReleaseEvent}

\end{description}\end{quote}

\end{fulllineitems}

\index{newTokenValue() (model.PlaceItem.PlaceItem method)}

\begin{fulllineitems}
\phantomsection\label{model_link:model.PlaceItem.PlaceItem.newTokenValue}\pysiglinewithargsret{\bfcode{newTokenValue}}{}{}
Accept new token.

\end{fulllineitems}

\index{renameModifications() (model.PlaceItem.PlaceItem method)}

\begin{fulllineitems}
\phantomsection\label{model_link:model.PlaceItem.PlaceItem.renameModifications}\pysiglinewithargsret{\bfcode{renameModifications}}{\emph{name}}{}
Make neccessary modification and renaming.
\begin{quote}\begin{description}
\item[{Parameters}] \leavevmode
\textbf{\texttt{name}} -- Name of the CPN element assigned by the user.

\end{description}\end{quote}

\end{fulllineitems}

\index{setPort() (model.PlaceItem.PlaceItem method)}

\begin{fulllineitems}
\phantomsection\label{model_link:model.PlaceItem.PlaceItem.setPort}\pysiglinewithargsret{\bfcode{setPort}}{\emph{loadFromFile=False}}{}
Create a new \titleref{model.PortItem} or edit the existing port.
\begin{quote}\begin{description}
\item[{Parameters}] \leavevmode
\textbf{\texttt{loadFromFile}} -- Flag determining, whether the place is loaded from file.

\end{description}\end{quote}

\end{fulllineitems}

\index{setPortDiag() (model.PlaceItem.PlaceItem method)}

\begin{fulllineitems}
\phantomsection\label{model_link:model.PlaceItem.PlaceItem.setPortDiag}\pysiglinewithargsret{\bfcode{setPortDiag}}{}{}
Callback function to modify the port.

\end{fulllineitems}

\index{stackTokens() (model.PlaceItem.PlaceItem method)}

\begin{fulllineitems}
\phantomsection\label{model_link:model.PlaceItem.PlaceItem.stackTokens}\pysiglinewithargsret{\bfcode{stackTokens}}{}{}
Method to order the visual stacking of different tokens.

\end{fulllineitems}


\end{fulllineitems}

\index{bin\_() (in module model.PlaceItem)}

\begin{fulllineitems}
\phantomsection\label{model_link:model.PlaceItem.bin_}\pysiglinewithargsret{\code{model.PlaceItem.}\bfcode{bin\_}}{\emph{QTextStream}}{{ $\rightarrow$ QTextStream}}
\end{fulllineitems}

\index{hex\_() (in module model.PlaceItem)}

\begin{fulllineitems}
\phantomsection\label{model_link:model.PlaceItem.hex_}\pysiglinewithargsret{\code{model.PlaceItem.}\bfcode{hex\_}}{\emph{QTextStream}}{{ $\rightarrow$ QTextStream}}
\end{fulllineitems}

\index{oct\_() (in module model.PlaceItem)}

\begin{fulllineitems}
\phantomsection\label{model_link:model.PlaceItem.oct_}\pysiglinewithargsret{\code{model.PlaceItem.}\bfcode{oct\_}}{\emph{QTextStream}}{{ $\rightarrow$ QTextStream}}
\end{fulllineitems}

\phantomsection\label{model_link:module-model.TransitionItem}\index{model.TransitionItem (module)}\index{TransitionItem (class in model.TransitionItem)}

\begin{fulllineitems}
\phantomsection\label{model_link:model.TransitionItem.TransitionItem}\pysiglinewithargsret{\strong{class }\code{model.TransitionItem.}\bfcode{TransitionItem}}{\emph{editor}, \emph{name}, \emph{position}, \emph{guardExpression=None}, \emph{uniqueName='t0'}, \emph{loadFromFile=False}, \emph{substitutionTransition=False}, \emph{subnet=None}}{}
Bases: \code{PyQt4.QtGui.QGraphicsRectItem}, {\hyperref[model_link:model.AbstractItem.AbstractItem]{\crossref{\code{model.AbstractItem.AbstractItem}}}}

CPN Transition element.
\begin{quote}\begin{description}
\item[{Member editor}] \leavevmode
\titleref{gui.DiagramEditor}. Editor to show in.

\item[{Member name}] \leavevmode
Name of the CPN element assigned by the user.

\item[{Member substitutionTransition}] \leavevmode
Flag indication a substitutaion transition.

\item[{Member uniqueName}] \leavevmode
Unique name of this transition used by the simulator.

\item[{Member transition}] \leavevmode
SNAKES transition node with id \titleref{uniqueName}, only if \titleref{substitutionTransition} is \titleref{False} else None.

\item[{Member subnet}] \leavevmode
Name of the subnet this transition is contained in.

\item[{Member position}] \leavevmode
Position of this item in the \titleref{gui.DiagramScene}.

\item[{Member enabled}] \leavevmode
Flag determining if this transition is enabled.

\item[{Member guardExpression}] \leavevmode
The guard expression for this transition.

\item[{Member subnetBorder}] \leavevmode
Second border indicating a substitution transition.

\item[{Member exceptionString}] \leavevmode
Exception string containing simulator errors.

\item[{Member modeString}] \leavevmode
Mode string containing the enabled modes for this transition

\end{description}\end{quote}
\index{\_\_init\_\_() (model.TransitionItem.TransitionItem method)}

\begin{fulllineitems}
\phantomsection\label{model_link:model.TransitionItem.TransitionItem.__init__}\pysiglinewithargsret{\bfcode{\_\_init\_\_}}{\emph{editor}, \emph{name}, \emph{position}, \emph{guardExpression=None}, \emph{uniqueName='t0'}, \emph{loadFromFile=False}, \emph{substitutionTransition=False}, \emph{subnet=None}}{}
Create transition item.
\begin{quote}\begin{description}
\item[{Parameters}] \leavevmode\begin{itemize}
\item {} 
\textbf{\texttt{editor}} -- \titleref{gui.DiagramEditor}. Editor to show in.

\item {} 
\textbf{\texttt{name}} -- Name of the CPN element assigned by the user.

\item {} 
\textbf{\texttt{position}} -- Position of this item in the \titleref{gui.DiagramScene}.

\item {} 
\textbf{\texttt{guardExpression}} -- The guard expression for this transition.

\item {} 
\textbf{\texttt{uniqueName}} -- Unique name of this transition used by the simulator.

\item {} 
\textbf{\texttt{loadFromFile}} -- Flag determining, whether the transition is loaded from file.

\item {} 
\textbf{\texttt{substitutionTransition}} -- Flag indication a substitutaion transition.

\item {} 
\textbf{\texttt{subnet}} -- Name of the subnet this transition is contained in.

\end{itemize}

\end{description}\end{quote}

\end{fulllineitems}

\index{acceptEditGuard() (model.TransitionItem.TransitionItem method)}

\begin{fulllineitems}
\phantomsection\label{model_link:model.TransitionItem.TransitionItem.acceptEditGuard}\pysiglinewithargsret{\bfcode{acceptEditGuard}}{}{}
Apply modified guard expression, after \titleref{gui.NameDialog}.

\end{fulllineitems}

\index{contextMenuEvent() (model.TransitionItem.TransitionItem method)}

\begin{fulllineitems}
\phantomsection\label{model_link:model.TransitionItem.TransitionItem.contextMenuEvent}\pysiglinewithargsret{\bfcode{contextMenuEvent}}{\emph{event}}{}
Generate Context menu on context menu event.
\begin{quote}\begin{description}
\item[{Parameters}] \leavevmode
\textbf{\texttt{event}} -- QContextMenuEvent.

\end{description}\end{quote}

\end{fulllineitems}

\index{deleteItemLocal() (model.TransitionItem.TransitionItem method)}

\begin{fulllineitems}
\phantomsection\label{model_link:model.TransitionItem.TransitionItem.deleteItemLocal}\pysiglinewithargsret{\bfcode{deleteItemLocal}}{}{}
Capture delete event and call editor delete function.

\end{fulllineitems}

\index{editGuard() (model.TransitionItem.TransitionItem method)}

\begin{fulllineitems}
\phantomsection\label{model_link:model.TransitionItem.TransitionItem.editGuard}\pysiglinewithargsret{\bfcode{editGuard}}{}{}
Callback function to modify the guard expression.

\end{fulllineitems}

\index{importSubnet() (model.TransitionItem.TransitionItem method)}

\begin{fulllineitems}
\phantomsection\label{model_link:model.TransitionItem.TransitionItem.importSubnet}\pysiglinewithargsret{\bfcode{importSubnet}}{}{}
Forward menu action \titleref{Import Subnet} to \titleref{gui.DiagramEditor}.

\end{fulllineitems}

\index{initTransition() (model.TransitionItem.TransitionItem method)}

\begin{fulllineitems}
\phantomsection\label{model_link:model.TransitionItem.TransitionItem.initTransition}\pysiglinewithargsret{\bfcode{initTransition}}{\emph{loadFromFile}}{}
Initialize transition.
\begin{quote}\begin{description}
\item[{Parameters}] \leavevmode
\textbf{\texttt{loadFromFile}} -- Flag determining, whether the transition is loaded from file.

\end{description}\end{quote}

\end{fulllineitems}

\index{mouseMoveEvent() (model.TransitionItem.TransitionItem method)}

\begin{fulllineitems}
\phantomsection\label{model_link:model.TransitionItem.TransitionItem.mouseMoveEvent}\pysiglinewithargsret{\bfcode{mouseMoveEvent}}{\emph{event}}{}
Prevents the movement of this item, when connections are drawn.
\begin{quote}\begin{description}
\item[{Parameters}] \leavevmode
\textbf{\texttt{event}} -- \titleref{QtGui.mouseMoveEvent}

\end{description}\end{quote}

\end{fulllineitems}

\index{mousePressEvent() (model.TransitionItem.TransitionItem method)}

\begin{fulllineitems}
\phantomsection\label{model_link:model.TransitionItem.TransitionItem.mousePressEvent}\pysiglinewithargsret{\bfcode{mousePressEvent}}{\emph{event}}{}
Prevents the movement of this item, when connections are drawn.
\begin{quote}\begin{description}
\item[{Parameters}] \leavevmode
\textbf{\texttt{event}} -- \titleref{QtGui.mousePressEvent}

\end{description}\end{quote}

\end{fulllineitems}

\index{mouseReleaseEvent() (model.TransitionItem.TransitionItem method)}

\begin{fulllineitems}
\phantomsection\label{model_link:model.TransitionItem.TransitionItem.mouseReleaseEvent}\pysiglinewithargsret{\bfcode{mouseReleaseEvent}}{\emph{event}}{}
Forward \titleref{QtGui.mouseReleaseEvent}.
\begin{quote}\begin{description}
\item[{Parameters}] \leavevmode
\textbf{\texttt{event}} -- \titleref{QtGui.mouseReleaseEvent}

\end{description}\end{quote}

\end{fulllineitems}

\index{openSubnet() (model.TransitionItem.TransitionItem method)}

\begin{fulllineitems}
\phantomsection\label{model_link:model.TransitionItem.TransitionItem.openSubnet}\pysiglinewithargsret{\bfcode{openSubnet}}{}{}
Forward menu action \titleref{Open Subnet} to \titleref{gui.DiagramEditor}.

\end{fulllineitems}

\index{renameModifications() (model.TransitionItem.TransitionItem method)}

\begin{fulllineitems}
\phantomsection\label{model_link:model.TransitionItem.TransitionItem.renameModifications}\pysiglinewithargsret{\bfcode{renameModifications}}{\emph{name}}{}
Make neccessary modification and renaming.
\begin{quote}\begin{description}
\item[{Parameters}] \leavevmode
\textbf{\texttt{name}} -- Name of the CPN element assigned by the user.

\end{description}\end{quote}

\end{fulllineitems}

\index{setInfoString() (model.TransitionItem.TransitionItem method)}

\begin{fulllineitems}
\phantomsection\label{model_link:model.TransitionItem.TransitionItem.setInfoString}\pysiglinewithargsret{\bfcode{setInfoString}}{\emph{stringInfo}}{}
Set the string for the \titleref{gui.AbstractItem.DescritionCanvas}
\begin{quote}\begin{description}
\item[{Parameters}] \leavevmode
\textbf{\texttt{stringInfo}} -- String appended to the first line of its \titleref{gui.AbstractItem.DescritionCanvas}.

\end{description}\end{quote}

\end{fulllineitems}


\end{fulllineitems}

\index{bin\_() (in module model.TransitionItem)}

\begin{fulllineitems}
\phantomsection\label{model_link:model.TransitionItem.bin_}\pysiglinewithargsret{\code{model.TransitionItem.}\bfcode{bin\_}}{\emph{QTextStream}}{{ $\rightarrow$ QTextStream}}
\end{fulllineitems}

\index{hex\_() (in module model.TransitionItem)}

\begin{fulllineitems}
\phantomsection\label{model_link:model.TransitionItem.hex_}\pysiglinewithargsret{\code{model.TransitionItem.}\bfcode{hex\_}}{\emph{QTextStream}}{{ $\rightarrow$ QTextStream}}
\end{fulllineitems}

\index{oct\_() (in module model.TransitionItem)}

\begin{fulllineitems}
\phantomsection\label{model_link:model.TransitionItem.oct_}\pysiglinewithargsret{\code{model.TransitionItem.}\bfcode{oct\_}}{\emph{QTextStream}}{{ $\rightarrow$ QTextStream}}
\end{fulllineitems}

\phantomsection\label{model_link:module-model.ArcItem}\index{model.ArcItem (module)}\index{ArcItem (class in model.ArcItem)}

\begin{fulllineitems}
\phantomsection\label{model_link:model.ArcItem.ArcItem}\pysiglinewithargsret{\strong{class }\code{model.ArcItem.}\bfcode{ArcItem}}{\emph{editor}, \emph{srcConnector}, \emph{dstConnector}, \emph{name='undefined'}, \emph{isPortConnection=False}}{}
Bases: \code{PyQt4.QtGui.QGraphicsItem}

CPN arc connection element.
\begin{quote}\begin{description}
\item[{Member editor}] \leavevmode
\titleref{gui.DiagramEditor}. Editor to show in.

\item[{Member srcConnector}] \leavevmode
Source \titleref{model.AbstractItem.Connector}.

\item[{Member dstConnector}] \leavevmode
Destination \titleref{model.AbstractItem.Connector}.

\item[{Member isPortConnection}] \leavevmode
Flag that determines, whether the arc represents the connection between a substitution transition and a port place..

\item[{Member name}] \leavevmode
Expression of the CPN arc assigned by the user.

\item[{Member arcDefined}] \leavevmode
Flag that determines, whether an arc creation was successful (internal).

\item[{Member variable}] \leavevmode
SNAKES Variable.

\item[{Member expression}] \leavevmode
SNAKES Expression.

\item[{Member pos1}] \leavevmode
Point of origin of the arc.

\item[{Member pos2}] \leavevmode
Point of destination of the arc.

\item[{Member arrowPolygon}] \leavevmode
\titleref{QtGui.QPolygonF} showing the direction of the arc.

\end{description}\end{quote}
\index{\_\_init\_\_() (model.ArcItem.ArcItem method)}

\begin{fulllineitems}
\phantomsection\label{model_link:model.ArcItem.ArcItem.__init__}\pysiglinewithargsret{\bfcode{\_\_init\_\_}}{\emph{editor}, \emph{srcConnector}, \emph{dstConnector}, \emph{name='undefined'}, \emph{isPortConnection=False}}{}~\begin{quote}\begin{description}
\item[{Parameters}] \leavevmode\begin{itemize}
\item {} 
\textbf{\texttt{editor}} -- \titleref{gui.DiagramEditor}. Editor to show in.

\item {} 
\textbf{\texttt{srcConnector}} -- Source \titleref{model.AbstractItem.Connector}.

\item {} 
\textbf{\texttt{dstConnector}} -- Destination \titleref{model.AbstractItem.Connector}.

\item {} 
\textbf{\texttt{name}} -- Expression of the CPN arc assigned by the user.

\item {} 
\textbf{\texttt{isPortConnection}} -- Flag that determines, whether the arc represents the connection between a substitution transition and a port place..

\end{itemize}

\end{description}\end{quote}

\end{fulllineitems}

\index{checkForExpression() (model.ArcItem.ArcItem method)}

\begin{fulllineitems}
\phantomsection\label{model_link:model.ArcItem.ArcItem.checkForExpression}\pysiglinewithargsret{\bfcode{checkForExpression}}{\emph{text}}{}
Check whether the annotation is intended to be an expression or a variable.
\begin{quote}\begin{description}
\item[{Parameters}] \leavevmode
\textbf{\texttt{text}} -- Annotation text.

\item[{Return ret}] \leavevmode
True/False.

\end{description}\end{quote}

\end{fulllineitems}

\index{deleteItemLocal() (model.ArcItem.ArcItem method)}

\begin{fulllineitems}
\phantomsection\label{model_link:model.ArcItem.ArcItem.deleteItemLocal}\pysiglinewithargsret{\bfcode{deleteItemLocal}}{}{}
Capture delete event and call editor delete function.

\end{fulllineitems}

\index{getName() (model.ArcItem.ArcItem method)}

\begin{fulllineitems}
\phantomsection\label{model_link:model.ArcItem.ArcItem.getName}\pysiglinewithargsret{\bfcode{getName}}{}{}
Return the annotation of this arc.

\end{fulllineitems}

\index{multiInput() (model.ArcItem.ArcItem method)}

\begin{fulllineitems}
\phantomsection\label{model_link:model.ArcItem.ArcItem.multiInput}\pysiglinewithargsret{\bfcode{multiInput}}{\emph{multiArcAnnotations}}{}
Register a multi input arc in the simulator.
\begin{quote}\begin{description}
\item[{Parameters}] \leavevmode
\textbf{\texttt{multiArcAnnotations}} -- SNAKES MultiArc annotation.

\end{description}\end{quote}

\end{fulllineitems}

\index{rename() (model.ArcItem.ArcItem method)}

\begin{fulllineitems}
\phantomsection\label{model_link:model.ArcItem.ArcItem.rename}\pysiglinewithargsret{\bfcode{rename}}{}{}
Rename arc and apply changes to simulator and visual representation.

\end{fulllineitems}

\index{setArcAnnotation() (model.ArcItem.ArcItem method)}

\begin{fulllineitems}
\phantomsection\label{model_link:model.ArcItem.ArcItem.setArcAnnotation}\pysiglinewithargsret{\bfcode{setArcAnnotation}}{\emph{annotationText=None}}{}
Finalize arc creation and set annotation.
\begin{quote}\begin{description}
\item[{Parameters}] \leavevmode
\textbf{\texttt{annotationText}} -- String arc annotation.

\item[{Return self.arcDefined}] \leavevmode
Flag that determines, whether an arc creation was successful (internal).

\end{description}\end{quote}

\end{fulllineitems}

\index{setBeginPos() (model.ArcItem.ArcItem method)}

\begin{fulllineitems}
\phantomsection\label{model_link:model.ArcItem.ArcItem.setBeginPos}\pysiglinewithargsret{\bfcode{setBeginPos}}{\emph{pos1}}{}
Callback method to keep \titleref{pos1} up to date.
\begin{quote}\begin{description}
\item[{Parameters}] \leavevmode
\textbf{\texttt{pos1}} -- \titleref{QtCore.QPointF()}.

\end{description}\end{quote}

\end{fulllineitems}

\index{setDestination() (model.ArcItem.ArcItem method)}

\begin{fulllineitems}
\phantomsection\label{model_link:model.ArcItem.ArcItem.setDestination}\pysiglinewithargsret{\bfcode{setDestination}}{\emph{dstConnector}}{}
Sets the destination \titleref{model.AbstractItem.Connector}.
\begin{quote}\begin{description}
\item[{Parameters}] \leavevmode
\textbf{\texttt{dstConnector}} -- Destination \titleref{model.AbstractItem.Connector}.

\end{description}\end{quote}

\end{fulllineitems}

\index{setEndPos() (model.ArcItem.ArcItem method)}

\begin{fulllineitems}
\phantomsection\label{model_link:model.ArcItem.ArcItem.setEndPos}\pysiglinewithargsret{\bfcode{setEndPos}}{\emph{endpos}}{}
Callback method to keep \titleref{pos2} up to date.
\begin{quote}\begin{description}
\item[{Parameters}] \leavevmode
\textbf{\texttt{endpos}} -- \titleref{QtCore.QPointF()}.

\end{description}\end{quote}

\end{fulllineitems}

\index{setName() (model.ArcItem.ArcItem method)}

\begin{fulllineitems}
\phantomsection\label{model_link:model.ArcItem.ArcItem.setName}\pysiglinewithargsret{\bfcode{setName}}{\emph{name}}{}
Set the annotation of this arc.
\begin{quote}\begin{description}
\item[{Parameters}] \leavevmode
\textbf{\texttt{name}} -- New annotation.

\end{description}\end{quote}

\end{fulllineitems}

\index{setPolygon() (model.ArcItem.ArcItem method)}

\begin{fulllineitems}
\phantomsection\label{model_link:model.ArcItem.ArcItem.setPolygon}\pysiglinewithargsret{\bfcode{setPolygon}}{}{}
Calculate position and rotation of the arc arrow head.

\end{fulllineitems}

\index{singleInput() (model.ArcItem.ArcItem method)}

\begin{fulllineitems}
\phantomsection\label{model_link:model.ArcItem.ArcItem.singleInput}\pysiglinewithargsret{\bfcode{singleInput}}{\emph{variableExpression}}{}
Register a single input arc in the simulator.
\begin{quote}\begin{description}
\item[{Parameters}] \leavevmode
\textbf{\texttt{variableExpression}} -- SNAKES Variable or Expression.

\end{description}\end{quote}

\end{fulllineitems}

\index{singleOutput() (model.ArcItem.ArcItem method)}

\begin{fulllineitems}
\phantomsection\label{model_link:model.ArcItem.ArcItem.singleOutput}\pysiglinewithargsret{\bfcode{singleOutput}}{\emph{variableExpression}}{}
Register a multi input arc in the simulator.
\begin{quote}\begin{description}
\item[{Parameters}] \leavevmode
\textbf{\texttt{variableExpression}} -- SNAKES Variable or Expression.

\end{description}\end{quote}

\end{fulllineitems}


\end{fulllineitems}

\index{LineItem (class in model.ArcItem)}

\begin{fulllineitems}
\phantomsection\label{model_link:model.ArcItem.LineItem}\pysiglinewithargsret{\strong{class }\code{model.ArcItem.}\bfcode{LineItem}}{\emph{parent}}{}
Bases: \code{PyQt4.QtGui.QGraphicsLineItem}

Visual representation of the line itself.
\begin{quote}\begin{description}
\item[{Member parent}] \leavevmode
Parent \titleref{model.ArcItem}.

\end{description}\end{quote}
\index{\_\_init\_\_() (model.ArcItem.LineItem method)}

\begin{fulllineitems}
\phantomsection\label{model_link:model.ArcItem.LineItem.__init__}\pysiglinewithargsret{\bfcode{\_\_init\_\_}}{\emph{parent}}{}
Create line.
\begin{quote}\begin{description}
\item[{Parameters}] \leavevmode
\textbf{\texttt{parent}} -- Parent \titleref{model.ArcItem}.

\end{description}\end{quote}

\end{fulllineitems}

\index{mouseDoubleClickEvent() (model.ArcItem.LineItem method)}

\begin{fulllineitems}
\phantomsection\label{model_link:model.ArcItem.LineItem.mouseDoubleClickEvent}\pysiglinewithargsret{\bfcode{mouseDoubleClickEvent}}{\emph{event}}{}
Edit arc annotation on \titleref{QtGui.mouseDoubleClickEvent}.
\begin{quote}\begin{description}
\item[{Parameters}] \leavevmode
\textbf{\texttt{event}} -- \titleref{QtGui.mouseDoubleClickEvent}.

\end{description}\end{quote}

\end{fulllineitems}


\end{fulllineitems}

\index{bin\_() (in module model.ArcItem)}

\begin{fulllineitems}
\phantomsection\label{model_link:model.ArcItem.bin_}\pysiglinewithargsret{\code{model.ArcItem.}\bfcode{bin\_}}{\emph{QTextStream}}{{ $\rightarrow$ QTextStream}}
\end{fulllineitems}

\index{hex\_() (in module model.ArcItem)}

\begin{fulllineitems}
\phantomsection\label{model_link:model.ArcItem.hex_}\pysiglinewithargsret{\code{model.ArcItem.}\bfcode{hex\_}}{\emph{QTextStream}}{{ $\rightarrow$ QTextStream}}
\end{fulllineitems}

\index{oct\_() (in module model.ArcItem)}

\begin{fulllineitems}
\phantomsection\label{model_link:model.ArcItem.oct_}\pysiglinewithargsret{\code{model.ArcItem.}\bfcode{oct\_}}{\emph{QTextStream}}{{ $\rightarrow$ QTextStream}}
\end{fulllineitems}

\phantomsection\label{model_link:module-model.TokenItem}\index{model.TokenItem (module)}\index{TokenItem (class in model.TokenItem)}

\begin{fulllineitems}
\phantomsection\label{model_link:model.TokenItem.TokenItem}\pysiglinewithargsret{\strong{class }\code{model.TokenItem.}\bfcode{TokenItem}}{\emph{editor}, \emph{token}, \emph{count}, \emph{qpos}, \emph{parent=None}}{}
Bases: \code{PyQt4.QtGui.QGraphicsEllipseItem}

A String Token.
\begin{quote}\begin{description}
\item[{Member editor}] \leavevmode
Referenced editor.

\item[{Member countToken}] \leavevmode
Number of tokens.

\item[{Member countTokenLabel}] \leavevmode
Visible representation of the number of tokens.

\item[{Member token}] \leavevmode
String value of token, shown in tooltip

\end{description}\end{quote}
\index{\_\_init\_\_() (model.TokenItem.TokenItem method)}

\begin{fulllineitems}
\phantomsection\label{model_link:model.TokenItem.TokenItem.__init__}\pysiglinewithargsret{\bfcode{\_\_init\_\_}}{\emph{editor}, \emph{token}, \emph{count}, \emph{qpos}, \emph{parent=None}}{}
Create a token.
\begin{quote}\begin{description}
\item[{Parameters}] \leavevmode\begin{itemize}
\item {} 
\textbf{\texttt{editor}} -- DiagramEditor. Editor to show in.

\item {} 
\textbf{\texttt{token}} -- Token value.

\item {} 
\textbf{\texttt{count}} -- Number of Tokens to create.

\item {} 
\textbf{\texttt{qpos}} -- Parent top right position.

\item {} 
\textbf{\texttt{parent=None}} -- Parent Place Element

\end{itemize}

\end{description}\end{quote}

\end{fulllineitems}

\index{contextMenuEvent() (model.TokenItem.TokenItem method)}

\begin{fulllineitems}
\phantomsection\label{model_link:model.TokenItem.TokenItem.contextMenuEvent}\pysiglinewithargsret{\bfcode{contextMenuEvent}}{\emph{event}}{}
Generate Context menu on context menu event.
\begin{quote}\begin{description}
\item[{Parameters}] \leavevmode
\textbf{\texttt{event}} -- QContextMenuEvent.

\end{description}\end{quote}

\end{fulllineitems}

\index{deleteItemLocal() (model.TokenItem.TokenItem method)}

\begin{fulllineitems}
\phantomsection\label{model_link:model.TokenItem.TokenItem.deleteItemLocal}\pysiglinewithargsret{\bfcode{deleteItemLocal}}{}{}
Capture delete event and call editor delete function.

\end{fulllineitems}

\index{setCountToken() (model.TokenItem.TokenItem method)}

\begin{fulllineitems}
\phantomsection\label{model_link:model.TokenItem.TokenItem.setCountToken}\pysiglinewithargsret{\bfcode{setCountToken}}{\emph{count}}{}
Token count, shown in green circle.
\begin{quote}\begin{description}
\item[{Parameters}] \leavevmode
\textbf{\texttt{count}} -- Number to show.

\end{description}\end{quote}

\end{fulllineitems}


\end{fulllineitems}

\index{bin\_() (in module model.TokenItem)}

\begin{fulllineitems}
\phantomsection\label{model_link:model.TokenItem.bin_}\pysiglinewithargsret{\code{model.TokenItem.}\bfcode{bin\_}}{\emph{QTextStream}}{{ $\rightarrow$ QTextStream}}
\end{fulllineitems}

\index{hex\_() (in module model.TokenItem)}

\begin{fulllineitems}
\phantomsection\label{model_link:model.TokenItem.hex_}\pysiglinewithargsret{\code{model.TokenItem.}\bfcode{hex\_}}{\emph{QTextStream}}{{ $\rightarrow$ QTextStream}}
\end{fulllineitems}

\index{oct\_() (in module model.TokenItem)}

\begin{fulllineitems}
\phantomsection\label{model_link:model.TokenItem.oct_}\pysiglinewithargsret{\code{model.TokenItem.}\bfcode{oct\_}}{\emph{QTextStream}}{{ $\rightarrow$ QTextStream}}
\end{fulllineitems}

\phantomsection\label{model_link:module-model.PortItem}\index{model.PortItem (module)}\index{PortItem (class in model.PortItem)}

\begin{fulllineitems}
\phantomsection\label{model_link:model.PortItem.PortItem}\pysiglinewithargsret{\strong{class }\code{model.PortItem.}\bfcode{PortItem}}{\emph{direction}, \emph{parent=None}}{}
Bases: \code{PyQt4.QtGui.QGraphicsEllipseItem}

Port place indicator.

The label indicates the port direction of the port place.
\begin{quote}\begin{description}
\item[{Member direction}] \leavevmode
Port direction: (i)nput, (o)utput, (io) bidirectional.

\item[{Member parent}] \leavevmode
Parent port place.

\item[{Member label}] \leavevmode
Visual representation of direction.

\end{description}\end{quote}
\index{\_\_init\_\_() (model.PortItem.PortItem method)}

\begin{fulllineitems}
\phantomsection\label{model_link:model.PortItem.PortItem.__init__}\pysiglinewithargsret{\bfcode{\_\_init\_\_}}{\emph{direction}, \emph{parent=None}}{}
Create a port
\begin{quote}\begin{description}
\item[{Parameters}] \leavevmode\begin{itemize}
\item {} 
\textbf{\texttt{direction}} -- Port direction: (i)nput, (o)utput, (io) bidirectional.

\item {} 
\textbf{\texttt{parent}} -- Parent port place.

\end{itemize}

\end{description}\end{quote}

\end{fulllineitems}

\index{contextMenuEvent() (model.PortItem.PortItem method)}

\begin{fulllineitems}
\phantomsection\label{model_link:model.PortItem.PortItem.contextMenuEvent}\pysiglinewithargsret{\bfcode{contextMenuEvent}}{\emph{event}}{}
Generate Context menu on context menu event.
\begin{quote}\begin{description}
\item[{Parameters}] \leavevmode
\textbf{\texttt{event}} -- QContextMenuEvent.

\end{description}\end{quote}

\end{fulllineitems}

\index{editPort() (model.PortItem.PortItem method)}

\begin{fulllineitems}
\phantomsection\label{model_link:model.PortItem.PortItem.editPort}\pysiglinewithargsret{\bfcode{editPort}}{}{}
Edit port direction.

\end{fulllineitems}

\index{getDirection() (model.PortItem.PortItem method)}

\begin{fulllineitems}
\phantomsection\label{model_link:model.PortItem.PortItem.getDirection}\pysiglinewithargsret{\bfcode{getDirection}}{}{}
Return direction of the port: ``i'', ``o'', ``io''.

\end{fulllineitems}

\index{itemChange() (model.PortItem.PortItem method)}

\begin{fulllineitems}
\phantomsection\label{model_link:model.PortItem.PortItem.itemChange}\pysiglinewithargsret{\bfcode{itemChange}}{\emph{change}, \emph{value}}{}
Item position has changed, calculate new position.
\begin{quote}\begin{description}
\item[{Parameters}] \leavevmode\begin{itemize}
\item {} 
\textbf{\texttt{change}} -- Change value.

\item {} 
\textbf{\texttt{value}} -- QtCore.QPointF().

\end{itemize}

\item[{Return value}] \leavevmode
QtCore.QPointF(x, y) or super(PortItem, self).itemChange(change, value).

\end{description}\end{quote}

\end{fulllineitems}

\index{setDirection() (model.PortItem.PortItem method)}

\begin{fulllineitems}
\phantomsection\label{model_link:model.PortItem.PortItem.setDirection}\pysiglinewithargsret{\bfcode{setDirection}}{\emph{direction}}{}
Set direction of the port.
\begin{quote}\begin{description}
\item[{Parameters}] \leavevmode
\textbf{\texttt{direction}} -- Direction of port: ``i'', ``o'', ``io''.

\end{description}\end{quote}

\end{fulllineitems}

\index{setPort() (model.PortItem.PortItem method)}

\begin{fulllineitems}
\phantomsection\label{model_link:model.PortItem.PortItem.setPort}\pysiglinewithargsret{\bfcode{setPort}}{}{}
Is this really neccessary?

\end{fulllineitems}


\end{fulllineitems}

\index{bin\_() (in module model.PortItem)}

\begin{fulllineitems}
\phantomsection\label{model_link:model.PortItem.bin_}\pysiglinewithargsret{\code{model.PortItem.}\bfcode{bin\_}}{\emph{QTextStream}}{{ $\rightarrow$ QTextStream}}
\end{fulllineitems}

\index{hex\_() (in module model.PortItem)}

\begin{fulllineitems}
\phantomsection\label{model_link:model.PortItem.hex_}\pysiglinewithargsret{\code{model.PortItem.}\bfcode{hex\_}}{\emph{QTextStream}}{{ $\rightarrow$ QTextStream}}
\end{fulllineitems}

\index{oct\_() (in module model.PortItem)}

\begin{fulllineitems}
\phantomsection\label{model_link:model.PortItem.oct_}\pysiglinewithargsret{\code{model.PortItem.}\bfcode{oct\_}}{\emph{QTextStream}}{{ $\rightarrow$ QTextStream}}
\end{fulllineitems}



\chapter{inout}
\label{inout_link:module-inout.XMLIO}\label{inout_link:inout}\label{inout_link::doc}\index{inout.XMLIO (module)}\index{XMLIO (class in inout.XMLIO)}

\begin{fulllineitems}
\phantomsection\label{inout_link:inout.XMLIO.XMLIO}\pysiglinewithargsret{\strong{class }\code{inout.XMLIO.}\bfcode{XMLIO}}{\emph{simulator}, \emph{rootElementName='`}}{}
Bases: \code{object}

XML Input and Output
\index{loadNet() (inout.XMLIO.XMLIO method)}

\begin{fulllineitems}
\phantomsection\label{inout_link:inout.XMLIO.XMLIO.loadNet}\pysiglinewithargsret{\bfcode{loadNet}}{\emph{filename}}{}
Parse XML file and prepare data for object creation.
\begin{quote}\begin{description}
\item[{Parameters}] \leavevmode
\textbf{\texttt{filename}} -- Filepath to XML file which shall be loaded.

\item[{Return {[} netName, subnets, serConnections, serPlaces, serTransitions {]}}] \leavevmode
A list containing lists for object creation.

\end{description}\end{quote}

\end{fulllineitems}

\index{netToXML() (inout.XMLIO.XMLIO method)}

\begin{fulllineitems}
\phantomsection\label{inout_link:inout.XMLIO.XMLIO.netToXML}\pysiglinewithargsret{\bfcode{netToXML}}{\emph{subnet}, \emph{placesS}, \emph{transitionsS}, \emph{connectionsS}}{}
Save Colored Petrinet \textbf{Subnet} to XML tree.
\begin{quote}\begin{description}
\item[{Parameters}] \leavevmode\begin{itemize}
\item {} 
\textbf{\texttt{subnet}} -- Subnet name.

\item {} 
\textbf{\texttt{placesS}} -- Place contained in \titleref{subnet}.

\item {} 
\textbf{\texttt{transitionsS}} -- Transitions contained in \titleref{subnet}.

\item {} 
\textbf{\texttt{connectionsS}} -- Connections contained in \titleref{subnet}.

\end{itemize}

\end{description}\end{quote}

\end{fulllineitems}

\index{saveLog() (inout.XMLIO.XMLIO method)}

\begin{fulllineitems}
\phantomsection\label{inout_link:inout.XMLIO.XMLIO.saveLog}\pysiglinewithargsret{\bfcode{saveLog}}{\emph{logList}}{}
Save Log Entries to XML tree.
\begin{quote}\begin{description}
\item[{Parameters}] \leavevmode
\textbf{\texttt{logList}} -- List of log entries from the log widget.

\end{description}\end{quote}

\end{fulllineitems}

\index{saveNet() (inout.XMLIO.XMLIO method)}

\begin{fulllineitems}
\phantomsection\label{inout_link:inout.XMLIO.XMLIO.saveNet}\pysiglinewithargsret{\bfcode{saveNet}}{\emph{filename}}{}
Save XML tree to file.
\begin{quote}\begin{description}
\item[{Parameters}] \leavevmode
\textbf{\texttt{filename}} -- Filepath where the XML tree is saved.

\end{description}\end{quote}

\end{fulllineitems}


\end{fulllineitems}

\index{bin\_() (in module inout.XMLIO)}

\begin{fulllineitems}
\phantomsection\label{inout_link:inout.XMLIO.bin_}\pysiglinewithargsret{\code{inout.XMLIO.}\bfcode{bin\_}}{\emph{QTextStream}}{{ $\rightarrow$ QTextStream}}
\end{fulllineitems}

\index{hex\_() (in module inout.XMLIO)}

\begin{fulllineitems}
\phantomsection\label{inout_link:inout.XMLIO.hex_}\pysiglinewithargsret{\code{inout.XMLIO.}\bfcode{hex\_}}{\emph{QTextStream}}{{ $\rightarrow$ QTextStream}}
\end{fulllineitems}

\index{oct\_() (in module inout.XMLIO)}

\begin{fulllineitems}
\phantomsection\label{inout_link:inout.XMLIO.oct_}\pysiglinewithargsret{\code{inout.XMLIO.}\bfcode{oct\_}}{\emph{QTextStream}}{{ $\rightarrow$ QTextStream}}
\end{fulllineitems}



\chapter{Indices and tables}
\label{index:indices-and-tables}\begin{itemize}
\item {} 
\DUrole{xref,std,std-ref}{genindex}

\item {} 
\DUrole{xref,std,std-ref}{modindex}

\item {} 
\DUrole{xref,std,std-ref}{search}

\end{itemize}


\renewcommand{\indexname}{Python Module Index}
\begin{theindex}
\def\bigletter#1{{\Large\sffamily#1}\nopagebreak\vspace{1mm}}
\bigletter{g}
\item {\texttt{gui.DiagramEditor}}, \pageref{gui_link:module-gui.DiagramEditor}
\item {\texttt{gui.DiagramScene}}, \pageref{gui_link:module-gui.DiagramScene}
\item {\texttt{gui.LibraryModel}}, \pageref{gui_link:module-gui.LibraryModel}
\item {\texttt{gui.MainWindow}}, \pageref{gui_link:module-gui.MainWindow}
\item {\texttt{gui.NameDialog}}, \pageref{gui_link:module-gui.NameDialog}
\item {\texttt{gui.ParameterDialog}}, \pageref{gui_link:module-gui.ParameterDialog}
\item {\texttt{gui.SubnetDialog}}, \pageref{gui_link:module-gui.SubnetDialog}
\item {\texttt{gui.TokenDialog}}, \pageref{gui_link:module-gui.TokenDialog}
\indexspace
\bigletter{i}
\item {\texttt{inout.XMLIO}}, \pageref{inout_link:module-inout.XMLIO}
\indexspace
\bigletter{m}
\item {\texttt{model.AbstractItem}}, \pageref{model_link:module-model.AbstractItem}
\item {\texttt{model.ArcItem}}, \pageref{model_link:module-model.ArcItem}
\item {\texttt{model.CPNSimulator}}, \pageref{model_link:module-model.CPNSimulator}
\item {\texttt{model.PlaceItem}}, \pageref{model_link:module-model.PlaceItem}
\item {\texttt{model.PortItem}}, \pageref{model_link:module-model.PortItem}
\item {\texttt{model.TokenItem}}, \pageref{model_link:module-model.TokenItem}
\item {\texttt{model.TransitionItem}}, \pageref{model_link:module-model.TransitionItem}
\end{theindex}

\renewcommand{\indexname}{Index}
\printindex
\end{document}
